\textit{1.}
设\(T \in L(U,V),S \in L(V,W)\)均可逆.
求证:\(ST \in L(U,W)\)可逆,且\((ST)^{-1}=T^{-1}S^{-1}\).

\textit{Proof}:
先证明\(ST\)是单射变换.令\((ST)(u)=0\),则\(S(Tu)=0\).

而\(\Ker S=\{0\}\),故\(Tu=0\).由于\(\Ker T=\{0\}\),故\(u=0\),得\(\Ker ST=\{0\}\),证毕.

再证明\(ST\)是满射变换.
由于\(\Img S=W\),故对于\(\forall w \in W\),\(\exists v \in V\),使得\(Sv=w\).

又由于\(\Img T=V\),故对于\(\forall v \in V\),\(\exists u \in U\),使得\(Tu=v\).

因此,\(\forall w \in W\),\(\exists u \in U\),使得\((ST)u=S(Tu)=Sv=w\),即\(\Img ST=W\).

可得\(ST\)是满射变换,因此\(ST\)是可逆变换,证毕.

再证\((ST)^{-1}=T^{-1}S^{-1}\).两边同乘\(ST\),得
    \begin{align*}
        (ST)(T^{-1}S^{-1})=S(TT^{-1})S^{-1}=SS^{-1}=I=(ST)(ST)^{-1}
    \end{align*}

\hspace*{\fill}

\textit{2.}
设\(V\)是有限维向量空间且\(\mydim V \geq 1\).

求证\(L(V)\)中所有不可逆算子构成的集合不是\(L(V)\)的一个子空间.

\textit{Proof}:
设\(v_1,\cdots,v_m\)是\(V\)的一组基.定义
    \begin{align*}
        T_1v_1=0,T_1v_i=v_i,i=2,\cdots,m \quad
        T_2v_i=v_i,T_2v_m=0,i=1,\cdots,m-1
    \end{align*}
很显然\(T_1,T_2\)都不是可逆变换,但是\(T_1+T_2\)是一个可逆变换,证毕.

\hspace*{\fill}

\textit{3.}
设\(V\)是有限维向量空间且\(U\)是\(V\)的一个子空间,\(S \in L(U,V)\).

求证:\(S\)是单射变换的充要条件是\(\exists T \in L(V)\),使得\(\forall u \in U,Tu=Su\).

\textit{Proof}:
必要性的证明是显然的.

由于\(T\)是单射变换,而\(\forall u \in U,Tu=Su\),故\(S\)也是单射变换.

充分性:设\(u_1,\cdots,u_m\)是\(U\)的一组基.

故存在线性无关的\(v_1,\cdots,v_n\),使得\(u_1,\cdots,u_m,v_1,\cdots,v_n\)是\(V\)的一组基.

由于\(S\)是单射变换,根据\textit{3.B.9},\(Su_1,\cdots,Su_m\)在\(V\)中线性无关.

故存在线性无关的\(w_1,\cdots,w_n\), 使得\(Su_1,\cdots,Su_m,w_1,\cdots,w_n\)是\(V\)的一组基.

定义线性算子\(T\).
    \begin{align*}
        Tu_i=Su_i,i=1,\cdots,m \quad Tv_j=w_j,j=1,\cdots,n
    \end{align*}
很显然\(T\)是单射变换且\(T \in L(V)\),故\(T\)是可逆变换.下证\(Tu=Su\).
    \begin{align*}
        \forall u=\sum_{i=1}^m a_iu_i,Tu=T\sum_{i=1}^m a_iu_i
        =\sum_{i=1}^m a_iTu_i=\sum_{i=1}^m a_iSu_i=S\sum_{i=1}^m a_iu_i=Su
    \end{align*}

\newpage

\textit{4.}
设\(W\)是有限维向量空间且\(T_1,T_2 \in L(V,W)\).

求证:\(\Ker T_1=\Ker T_2\)的充要条件是存在可逆算子\(S \in L(W)\),使得\(T_1=ST_2\).

\textit{Proof}:
必要性:\(\forall \mu \in \Ker T_1,T_1 \mu=ST_2 \mu =0\).

由于\(S\)是单射变换,故\(ST_2 \mu =0 \Rightarrow T_2 \mu =0\),即\(\mu \in \Ker T_2\).

综上,\(\forall \mu \in \Ker T_1, \mu \in \Ker T_2\),即\(\Ker T_1 \subseteq \Ker T_2\).

同时由于\(S\)是可逆变换,故而可以相同方法证明\(\Ker T_2 \subseteq \Ker T_1\),得到\(\Ker T_1=\Ker T_2\).

充分性:\(W\)是有限维向量空间且\(\Img T_2 \subseteq W\),故\(\Img T_2\)也是有限维向量空间.

令\(T_2v_1,\cdots,T_2v_m\)是\(\Img T_2\)的一组基.根据\textit{3.B.24}和\textit{3.A.4},

令\(K=\myspan (v_1,\cdots,v_m)\),则\(V=\Ker T \oplus K\),且\(v_1,\cdots,v_m\)线性无关.

下证\(T_1v_1,\cdots,T_1v_m\)也线性无关.令\(\sum_{i=1}^m a_iT_1v_i=T_1\sum_{i=1}^m a_iv_i=0\).

由于\(\Ker T_1=\Ker T_2\)且\(\Ker T \cap \myspan (v_1,\cdots,v_m)=\{0\}\),故\(\sum_{i=1}^m a_iv_i=0\).

结合\(v_1,\cdots,v_m\)线性无关得到\(a_1=\cdots=a_m=0\),即\(T_1v_1,\cdots,T_1v_m\)也线性无关.

现在将\(T_2v_1,\cdots,T_2v_m\)和\(T_1v_1,\cdots,T_1v_m\)分别补充为\(W\)的一组基

\(T_2v_1,\cdots,T_2v_m,w_1^\alpha,\cdots,w_n^\alpha\),其中\(w_1^\alpha,\cdots,w_n^\alpha\)线性无关;

\(T_1v_1,\cdots,T_1v_m,w_1^\beta,\cdots,w_n^\beta\),其中\(w_1^\beta,\cdots,w_n^\beta\)线性无关.

现在定义线性变换\(S\).令
    \begin{align*}
        S(T_2v_i)=T_1v_i,i=1,\cdots,m \quad Sw_j^\alpha=Sw_j^\beta,j=1,\cdots,n
    \end{align*}
定理3.5保证了线性变换的存在性.

现在证明\(S\)是单射变换.
令\(S(\sum_{i=1}^m a_iT_2v_i+\sum_{j=1}^n b_jw_j^\alpha)=\sum_{i=1}^m a_iT_1v_i+\sum_{j=1}^n b_jw_j^\beta=0\).

由\(T_1v_1,\cdots,T_1v_m,w_1^\beta,\cdots,w_n^\beta\)线性无关,故\(a_1=\cdots=a_m=b_1=\cdots=b_n=0\),

即\(\Ker S=\{0\}\).结合\(S \in L(W)\),根据定理3.69,\(S\)是可逆变换.

下证\(T_1=ST_2\).由于\(V=\Ker T \oplus K\),故\(\forall v \in V,v=v_0+\sum_{i=1}^m a_iv_i\),其中\(v_0 \in \Ker T\).
    \begin{align*}
        (ST_2)v&=(ST_2)(v_0+\sum_{i=1}^m a_iv_i)=\sum_{i=1}^m a_iS(T_2v_i)=\sum_{i=1}^m a_iT_1v_i \\
            &=T_1\sum_{i=1}^m a_iv_i=T_1(\sum_{i=1}^m a_iv_i+v_0)=T_1v
    \end{align*}
从而对于\(\forall v \in V\),\(T_1v=(ST_2)v\),即\(T_1=ST_2\),证毕.

\newpage

\textit{5.}
设\(V\)是有限维向量空间且\(T_1,T_2 \in L(V,W)\).

求证:\(\Img T_1=\Img T_2\)的充要条件是存在可逆算子\(S \in L(V)\),使得\(T_1=T_2S\).

\textit{Proof}:
必要性:\(\forall \mu \in V\),\(T_1 \mu =T_2S \mu \in \Img T_2\),即\(\Img T_1 \subseteq \Img T_2\).

又\(T_1=T_2S \Rightarrow T_1S^{-1}=T_2\),
故\(\forall \mu \in V\),\(T_2 \mu =T_1S^{-1} \mu \in \Img T_1\),即\(\Img T_2 \subseteq \Img T_1\).

综上\(\Img T_1=\Img T_2\),证毕.

充分性:令\(w_1,\cdots,w_m\)为\(\Img T\)的一组基.

找到\(v_1^\alpha,\cdots,v_m^\alpha\)和\(v_1^\beta,\cdots,v_m^\beta\),
使得\(\forall i=1,\cdots,m,T_1v_i^\alpha=w_i,T_2v_i^\beta=w_i\).

根据\textit{3.A.4},
\(v_1^\alpha,\cdots,v_m^\alpha\)和\(v_1^\beta,\cdots,v_m^\beta\)分别线性无关.

现在将\(v_1^\alpha,\cdots,v_m^\alpha\)和\(v_1^\beta,\cdots,v_m^\beta\)分别补充成\(V\)的一组基

\(v_1^\alpha,\cdots,v_m^\alpha,u_1^\alpha,\cdots,u_n^\alpha\),其中\(u_1^\alpha,\cdots,u_n^\alpha\)线性无关;

\(v_1^\beta,\cdots,v_m^\beta,u_1^\beta,\cdots,u_n^\beta\),其中\(u_1^\beta,\cdots,u_n^\beta\)线性无关.

根据\textit{3.B.24},
\(\Ker T_1=\myspan (u_1^\alpha,\cdots,u_m^\alpha),\Ker T_2=\myspan (u_1^\beta,\cdots,u_n^\beta)\).

现在定义线性变换\(S\).令
    \begin{align*}
        Sv_i^\beta=v_i^\alpha,i=1,\cdots,m \quad
        Su_j^\beta=u_j^\alpha,j=1,\cdots,n
    \end{align*}
定理3.5保证了线性变换的存在性.

现在证明\(S\)是单射变换.
令\(S(\sum_{i=1}^m a_iv_i^\alpha+\sum_{j=1}^n b_iu_i^\alpha)=\sum_{i=1}^m a_iv_i^\beta+\sum_{j=1}^n b_iu_i^\beta=0\).

由\(v_1^\beta,\cdots,v_m^\beta,u_1^\beta,\cdots,u_n^\beta\)线性无关,故\(a_1=\cdots=a_m=b_1=\cdots=b_n=0\),

即\(\Ker S=\{0\}\).结合\(S \in L(V)\),根据定理3.69,\(S\)是可逆变换.

下证\(T_1=T_2S\).\(\forall v=\sum_{i=1}^m a_iv_i^\alpha+\sum_{j=1}^n b_iu_i^\alpha \in V\),有
    \begin{align*}
        (T_2S)v&=(T_2S)(\sum_{i=1}^m a_iv_i^\alpha+\sum_{j=1}^n b_iu_i^\alpha )
                =T_2(\sum_{i=1}^m a_iv_i^\beta+\sum_{j=1}^n b_iu_i^\beta ) \\
            &=\sum_{i=1}^m a_iT_2v_i^\beta=T_2\sum_{i=1}^m a_iv_i^\beta
                =\sum_{i=1}^m a_iw_i=\sum_{i=1}^m a_iT_1v_i^\alpha \\
            &=T_1(\sum_{i=1}^m a_iv_i^\alpha+\sum_{j=1}^n b_iu_i^\alpha)=T_1v
    \end{align*}
从而对于\(\forall v \in V\),\(T_1v=(T_2S)v\),即\(T_1=T_2S\),证毕.

\newpage

\textit{6.}
设\(V\)和\(W\)是有限维向量空间且\(T_1,T_2 \in L(V,W)\).

求证:\(\mydim \Ker T_1=\mydim \Ker T_2\)等价于存在可逆算子\(R \in L(V),S \in L(W)\),使得\(T_1=ST_2R\).

\textit{Proof}:
必要性:\(T_1=ST_2R \Rightarrow S^{-1}T_1=T_2R\).

根据\textit{3.D.4},\(\Ker T_1=\Ker T_2R\).
根据\textit{3.D.5},\(\Img S^{-1}T_1=\Img T_2\).
    \begin{align*}
        \mydim \Ker T_1 &=\mydim \Ker T_2R=\mydim V-\mydim \Img T_2R \\
        &=\mydim V-\mydim \Img T_2=\mydim \Ker T_2
    \end{align*}

充分性:令\(v_1^\alpha,\cdots,v_m^\alpha\)和\(v_1^\beta,\cdots,v_m^\beta\)
分别为\(\Ker T_1\)和\(\Ker T_2\)的一组基.

将\(v_1^\alpha,\cdots,v_m^\alpha\)和\(v_1^\beta,\cdots,v_m^\beta\)分别补充成\(V\)的一组基

\(v_1^\alpha,\cdots,v_m^\alpha,u_1^\alpha,\cdots,u_n^\alpha\),
其中\(u_1^\alpha,\cdots,u_n^\alpha\)线性无关;

\(v_1^\beta,\cdots,v_m^\beta,u_1^\beta,\cdots,u_n^\beta\),
其中\(u_1^\beta,\cdots,u_n^\beta\)线性无关.

根据\textit{3.B.9}和\textit{3.B.12},
\(T_1u_1^\alpha,\cdots,T_1u_n^\alpha\)是\(\Img T_1\)的一组基,
\(T_2u_1^\beta,\cdots,T_2u_n^\beta\)是\(\Img T_2\)的一组基.

由于\(\Img T_1,\Img T_2 \subseteq W\),
将\(T_1u_1^\alpha,\cdots,T_1u_n^\alpha\)和\(T_2u_1^\beta,\cdots,T_2u_n^\beta\)分别补充成\(W\)的一组基

\(T_1u_1^\alpha,\cdots,T_1u_n^\alpha,w_1^\alpha,\cdots,w_p^\alpha\),
其中\(w_1^\alpha,\cdots,w_p^\alpha\)线性无关;

\(T_2u_1^\beta,\cdots,T_2u_n^\beta,w_1^\beta,\cdots,w_p^\beta\),
其中\(w_1^\beta,\cdots,w_p^\beta\)线性无关.

现在定义线性变换\(R\)和\(S\).令
    \begin{align*}
        Rv_i^\alpha=v_i^\beta,i=1,\cdots,m &\quad Ru_j^\alpha=u_j^\beta,j=1,\cdots,n \\
        S(T_2u_j^\beta)=T_1u_j^\alpha,j=1,\cdots,m &\quad Sw_k^\beta=w_k^\alpha,k=1,\cdots,p
    \end{align*}
参考\textit{3.D.4}和\textit{3.D.5}的证明,显然\(R\)和\(S\)都是可逆变换.

下证\(T_1=ST_2R\).\(\forall v=\sum_{i=1}^m a_iv_i^\alpha+\sum_{j=1}^n b_ju_j^\alpha\),
    \begin{align*}
        (ST_2R)v &=(ST_2R)(\sum_{i=1}^m a_iv_i^\alpha+\sum_{j=1}^n b_ju_j^\alpha)
                =(ST_2)(\sum_{i=1}^m a_iRv_i^\alpha+\sum_{j=1}^n b_jRu_j^\alpha) \\
                &=(ST_2)(\sum_{i=1}^m a_iv_i^\beta+\sum_{j=1}^n b_ju_j^\beta)
                =S(\sum_{i=1}^m a_iT_2v_i^\beta+\sum_{j=1}^n b_jT_2u_j^\beta) \\
                &=\sum_{j=1}^n b_jS(T_2u_j^\beta)=\sum_{j=1}^n b_jT_1u_j^\alpha
                =T_1(\sum_{i=1}^m a_iv_i^\alpha+\sum_{j=1}^n b_ju_j^\alpha)=T_1v
    \end{align*}
故\(\forall v \in V,T_1v=(ST_2R)v\),即\(T_1=ST_2R\),证毕.

\newpage

\textit{8.}
设\(V\)是有限维向量空间且\(T \in L(V,W)\)是一个满射变换.

求证:存在\(V\)的一个子空间\(U\),使得\(T|_U\)是一个可逆变换.

\textit{Proof}:
令\(w_1,\cdots,w_m\)为\(\Img T\)的一组基.找到\(v_1,\cdots,v_m\),使得\(\forall i=1,\cdots,m,T_1v_i=w_i\).

根据\textit{3.A.4},\(v_1,\cdots,v_m\)线性无关.令\(U=\myspan (v_1,\cdots,v_m)\).

下证\(T|_U\)是一个可逆变换.先证\(\Ker T|_U=\{0\}\).令\(Tv=T\sum_{i=1}^m a_iv_i=\sum_{i=1}^m a_iw_i=0\).

由于\(w_1,\cdots,w_m\)线性无关,故\(a_1=\cdots=a_m=0\),从而\(\Ker T|_U=\{0\}\),证毕.

又因为\(\Img T|_U=\Img T=\myspan (w_1,\cdots,w_m)\),故\(T|_U\)是满射变换.综上,\(T|_U\)是可逆变换.

\hspace*{\fill}

\textit{14.}
设\(v_1^,\cdots,v_n\)是\(V\)的一组基.定义\(T \in L(V,F^{n,1})\)为\(Tv=M(v)\).

求证:\(T\)是一个可逆变换.

\textit{Proof}:
\(\forall v \in V,\exists c_1,\cdots,c_n \in F\),使得\(v=\sum_{i=1}^n c_iv_i\).从而
    \begin{align*}
        T\sum_{i=1}^n c_iv_i=
            \begin{pmatrix}
                c_1 & \cdots & c_n
            \end{pmatrix}^T
    \end{align*}
下证\(T\)是单射变换.令\(M(v)=0\),则\(c_1=\cdots=c_n=0\),即\(v=0\),从而\(T\)是单射变换.

再证\(T\)是满射变换.\(\forall c_i \in F,\exists v \in V\),使得\(v=\sum_{i=1}^n c_iv_i\).因此\(T\)是满射变换,证毕.

\hspace*{\fill}

\textit{15.}
证明:若\(T \in L(F^{n,1},F^{m,1})\),则存在一个\(m-n\)矩阵\(A\),使得\(\forall x \in F^{n,1},Tx=Ax\).

\textit{Proof}:
使用\(F^{n,1}\)和\(F^{m,1}\)的标准基.

定义\(T\)为\(T(x_1,\cdots,x_n)=(\sum_{i=1}^m A_{1,i}x_i,\cdots,\sum _{i=1}^m A_{n,i}x_i)\).
在该定义下,矩阵\(A\)为
    \begin{equation*}
        \begin{pNiceMatrix}[first-row,first-col]
                &   x_1   & \cdots &  x_n     \\
        Tx_1   & A_{1,1} & \cdots & A_{1,n}  \\
        \vdots & \vdots  & \ddots & \vdots   \\
        Tx_n   & A_{m,1} & \cdots & A_{m,n}  \\
        \end{pNiceMatrix}
    \end{equation*}
因此总是存在这样的\(A_{i,j}\),证毕.

\newpage

\textit{17.}
设\(V\)是有限维向量空间且\(\varepsilon\)是\(L(V)\)的一个子空间,

其对所有\(S \in L(V)\)和\(T \in \varepsilon\)满足\(ST \in \varepsilon\)或\(TS \in L(V)\).
求证:\(\varepsilon=\{0\}\)或\(\varepsilon=L(V)\).

\textit{Proof}:
设\(e_1,\cdots,e_n\)是\(V\)的一组基.
当\(\varepsilon =\{0\}\)时,\(\forall ST=TS=0 \in \varepsilon\)是显然的;

若\(\varepsilon \ne \{0\}\),则\(\exists T \in \varepsilon, T \ne 0\).
从而\(\exists s,t \in \{1,\cdots,n\},Te_s=\sum_{i=1}^n a_ie_i ,Te_s \ne 0, a_t \ne 0\).

现在定义\(L(V)\)的一组基.令\(\varepsilon_{i,j}e_k=\delta_{i,k}e_j\),其中\(\delta_{i,k}=0,i \ne k,\delta_{i,k}=1,i=k\).

从而当\(\varepsilon_{i,j}\)中\(i,j\)分别取遍\(\{1,\cdots,n\}\)中每一个值时,\(\varepsilon_{i,j}\)成为\(L(V)\)的一组基.
    \begin{align*}
        \varepsilon_{t,i}T\varepsilon_{i,s}e_j=\varepsilon_{t,i}T(\delta_{i,j} e_s) 
        =\delta_{i,j}\varepsilon_{t,i}a_te_t=a_t\delta_{i,j}e_i
    \end{align*}
由于\(\varepsilon_{t,i}T\varepsilon_{i,s} \in \varepsilon\)且\(\varepsilon\)是\(L(V)\)的子空间,
故\(\sum_{i=1}^n \varepsilon_{t,i}T\varepsilon_{i,s}e_j=a_t \sum_{i=1}^n \delta_{i,j}e_i\).

由于只有\(i=j\)时,\(\delta_{i,j}=1\),因此除\(i=j\)的其它项被消去,
得\(\sum_{i=1}^n \varepsilon_{t,i} T \varepsilon_{i,s} e_j=a_te_j\).

即\(\sum_{i=1}^n \varepsilon_{t,i} T \varepsilon_{i,s} =a_t I\).
因此\(a_t I \in \varepsilon \Rightarrow I \in \varepsilon\).

故而\(\forall S \in L(V),S=SI \in \varepsilon\),即\(\varepsilon=L(V)\).

\hspace*{\fill}

\textit{19.}
设\(T \in L(P(R))\)满足\(T\)是单射变换,

且对于任意非零多项式\(p \in P(R)\)满足\(\mydeg Tp \leq \mydeg p\).

(\textit{a})证明:\(T\)是满射变换.

(\textit{b})证明:对于任意非零多项式\(p \in P(R)\)都有\(\mydeg Tp=\mydeg p\).

(\textit{a.Proof}):
由\(T \in L(P(R))\)单射结合\textit{3.D.3}得到\(\forall n \in N*,T|_{P_n(R)}\)单射.

结合定理3.69,\(T|_{P_n(R)}\)是一个满射变换,从而\(T\)也是一个满射变换.

(\textit{b.Proof}):
运用数学归纳法和反证法,设\(\mydeg p=n+1,\mydeg Tp<n+1\).

由\textit{3.D.19.a},\(T|_{P_n(R)}\)满射,故\(\exists q \in P_n(R),Tp=T|_{P_n(R)}q\).

由\(T\)为单射得\(p=q\),假设不成立,原命题得证.

