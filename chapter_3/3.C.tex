\textit{6.}
设\(V\)和\(W\)都是有限维向量空间且\(T \in L(V,W)\).

求证:\(\mydim \Img T=1\)等价于分别存在\(V\)和\(W\)的一组基,使得\(M(T)_{i,j}=1\).

\textit{Proof}
必要性:设\(v_1,\cdots,v_m\)和\(w_1,\cdots,w_n\)分别为\(V\)和\(W\)的一组基,

且这两组基可以使得\(M(T)\)中所有元素均为1.

从而有\(\forall i=1,\cdots,m,Tv_i=\sum_{i=1}^n w_i\),即\(\Img T=\myspan (\sum_{i=1}^m w_i),\mydim \Img T=1\),证毕.

充分性:设\(\mu_1,\cdots,\mu_m\)为\(V\)任意的一组基.

由于\(\mydim \Img T=1\),不妨设\(\Img T=\myspan (T \mu_1)\),即\(T \mu_2=\cdots=T \mu_m=0\).

则一定存在线性无关的\(w_2,\cdots,w_n\),使得\(T \mu_1,w_2,\cdots,w_n\)是\(W\)的一组基.

令\(w_1=T \mu_1-\sum_{i=2}^m w_i\),则\(w_1,w_2,\cdots,w_n\)也是\(W\)的一组基.

再令\(v_1=\mu_1,v_i=\mu_i+\mu_1,i=2,\cdots,m\),因而\(Tv_i=T(\mu_1+\mu_i)=T \mu_1=\sum_{i=1}^m w_i\).

从而\(v_1,\cdots,v_m\)和\(w_1,\cdots,w_m\)是满足条件的基.

\hspace*{\fill}

\textit{7.}
已知\(S,T \in L(V,W)\),求证:\(M(S+T)=M(S)+M(T)\).

\textit{Proof}
设\(v_1,\cdots,v_m\)和\(w_1,\cdots,w_n\)分别为\(V\)和\(W\)的一组基,并令\(M(S)=A,M(T)=C\).
    \begin{align*}
        &Sv_j=\sum_{i=1}^m A_{i,j}w_i,Tv_j=\sum_{i=1}^m C_{i,j}w_i \quad
        Sv_j+Tv_j=\sum_{i=1}^m (A_{i,j}+C_{i,j})w_i=(S+T)v_j \\
        &\forall j=1,\cdots,m,i=1,\cdots,n,(M(S+T))_{i,j}=M(S)_{i,j}+M(T)_{i,j}
    \end{align*}

即\(M(S+T)=M(S)+M(T)\),证毕.

\newpage

\textit{13.}
证明矩阵加法和乘法的分配律成立.

\textit{Proof}
即证明\(A(B+C)=AB+AC,(D+E)F=DF+EF\).

设\(A,D,E\)是\(m-n\)矩阵,\(B,C,F\)是\(n-p\)矩阵.
    \begin{align*}
        1^{\circ} &(A(B+C))_{i,j}=\sum_{k=1}^n A_{i,k}(B+C)_{k,j}=\sum_{k=1}^n A_{i,k}(B_{k,j}+C_{k,j}) \\
        &(AB+AC)_{i,j}=(AB)_{i,j}+(AC)_{i,j}=\sum_{k=1}^n A_{i,k}B_{k,j}+\sum_{k=1}^n A_{i,k}C_{k,j} \\
        &(A(B+C))_{i,j}=(AB+AC)_{i,j} \Rightarrow A(B+C)=AB+AC \\
        2^{\circ} &((D+E)F)_{i,j}=\sum_{k=1}^n (D+E)_{i,k}F_{k,j}=\sum_{k=1}^n (D_{i,k}+E_{i,k})F_{k,j} \\
        &(DF+EF)_{i,j}=(DF)_{i,j}+(EF)_{i,j}=\sum_{k=1}^n D_{i,k}F_{k,j}+\sum_{k=1}^n E_{i,k}F_{k,j} \\
        &((D+E)F)_{i,j}=(DF+EF)_{i,j} \Rightarrow (D+E)F=DF+EF
    \end{align*}

\hspace*{\fill}

\textit{15.}
设\(A\)是一个\(n-n\)矩阵,且\(1 \leq i,j \leq n\).
求证:\(A^3_{i,j}=\sum_{k=1}^n \sum_{p=1}^n A_{i,k}A_{k,p}A_{p,j}\).

\textit{Proof}
    \begin{align*}
        A^3_{i,j}=\sum_{p=1}^n A^2_{i,p}A_{p,j}
        =\sum_{p=1}^n (\sum_{k=1}^n A_{i,k}A_{k,p})A_{p,j}
        =\sum_{k=1}^n \sum_{p=1}^n A_{i,k}A_{k,p}A_{p,j}
    \end{align*}