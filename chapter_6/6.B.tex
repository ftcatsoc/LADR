\textit{Theorem 6.31 Gram-Schmidt}{\kaishu 正交化}

设\(v_1,\cdots,v_m\)是内积空间\(V\)中线性无关的向量组,构造正交向量组\(u_1,\cdots,u_m\).
    \begin{align*}
        u_1=v_1,\forall i=2,\cdots,m,u_i=v_i-\sum_{j=1}^{i-1} \dfrac{\ip{v_i}{u_j}}{\ip{u_j}{u_j}}u_j
    \end{align*}
令\(\forall i=1,\cdots,m,e_i=\dfrac{u_i}{\norm{u_i}}\),则\(e_1,\cdots,e_m\)规范正交.

\textit{Proof}
由于\(\forall i=1,\cdots,m,v_i \notin \myspan(v_1,\cdots,v_{i-1})\),故\(u_i \ne 0,\norm{u_i} \ne 0\).

因此\(\forall i=1,\cdots,m,\norm{e_i}=1\),故只需证明\(u_1,\cdots,u_m\)正交即可.

根据定理6.14呈现的正交分解,\(\forall j=1,\cdots,i-1,\ip{v_i-\dfrac{\ip{v_i}{u_j}}{\ip{u_j}{u_j}}u_j}{u_j}=0\).

因此\(\forall j=1,\cdots,i-1,\ip{u_i}{u_j}=0\).因此\(u_1,\cdots,u_m\)两两正交,证毕.

进一步,有\(\forall i=1,\cdots,m,v_i=u_i+\sum_{j=1}^{i-1} \dfrac{\ip{v_i}{u_j}}{\ip{u_j}{u_j}}u_j \in \myspan(u_1,\cdots,u_i)\).

从而\(\myspan(v_1,\cdots,v_i) \subseteq \myspan(u_1,\cdots,u_i)\).由于两组向量均线性无关且长度相等,

故\(\myspan(v_1,\cdots,v_i)=\myspan(u_1,\cdots,u_i)=\myspan(e_1,\cdots,e_i)\).

\hspace*{\fill}

\textit{Theorem 6.42 Risez}{\kaishu 表示定理}
设\(V\)是有限维内积空间且\(\varphi\)是\(V\)上的线性泛函.

求证:存在唯一的\(u \in V\),使得对于任意的\(v \in V\),都有\(\varphi(v)=\ip{v}{u}\).

\textit{Proof}
存在性:设\(e_1,\cdots,e_m\)是\(V\)的一组正交基,从而
    \begin{align*}
        \varphi(v)=\varphi(\sum_{i=1}^m \ip{v}{e_i}e_i)=\sum_{i=1}^m \ip{v}{e_i}\varphi(e_i)
        =\ip{v}{\sum_{i=1}^m \ol{\varphi(e_i)}e_i}=\ip{v}{u}
    \end{align*}
因此只需令\(u=\sum_{i=1}^m \ol{\varphi(e_i)}e_i\)即可,存在性得证.

唯一性:若存在\(u_1,\cdots,u_2 \in V\)满足\(\varphi(v)=\ip{v}{u_1}=\ip{v}{u_2}\),则
    \begin{align*}
        0=\ip{v}{u_1}-\ip{v}{u_2}=\ip{v}{u_1-u_2}
    \end{align*}
由\(V\)的任意性,取\(v=u_1-u_2\)时得到\(u_1-u_2=0,u_1=u_2\),唯一性得证.

\newpage

\textit{1*.} \footnote{标*的为第四版上的习题.}
设\(e_1,\cdots,e_m\)是\(V\)中的一向量组,使得\(\forall a_i \in F,\norm{\sum_{i=1}^m a_ie_i}^2=\sum_{i=1}^m \abs{a_i}^2\).

证明:\(e_1,\cdots,e_m\)是规范正交组.

\textit{Proof}
    \begin{align*}
        \norm{\sum_{i=1}^m a_ie_i}^2=\sum_{i=1}^m \sum_{i=1}^m \ip{a_ie_i}{a_je_j}=\sum_{i=1}^m \abs{a_i}^2
    \end{align*}
令\(\forall i \ne k,a_i=0\),可得\(\abs{a_{k}}^2\norm{e_{k}}^2=\abs{a_{k}}^2,\norm{e_{k}}=1\).
因此\(\forall i=1,\cdots,m,\norm{e_i}=1\).

令\(\forall i \ne j,k,a_i=0\),可得\(\ip{a_je_j}{a_ke_k}+\ip{a_ke_k}{a_je_j}=0\).

从而\(\mathrm{Re}(a_j\ol{a_k}\ip{e_j}{e_k})=0\).由\(a_j,a_k\)的任意性,必有\(\ip{e_j}{e_k}=0\).

从而\(e_1,\cdots,e_m\)是规范正交组,证毕.

\hspace*{\fill}

\textit{4.}
设\(n \in N^*\).求证:向量空间\(C[-\pi,\pi]\)中的
    \begin{align*}
        \dfrac{1}{\sqrt{2\pi}},\dfrac{\cos x}{\sqrt{\pi}},\cdots,\dfrac{\cos nx}{\sqrt{\pi}},
        \dfrac{\sin x}{\sqrt{\pi}},\cdots,\dfrac{\sin nx}{\sqrt{\pi}}
    \end{align*}
规范正交,其内积定义为\(\ip{f}{g}=\int_{-\pi}^{\pi}f(x)g(x)dx\).

\textit{Proof}
先证明其范数均为\(1\).
    \begin{equation*}
        \begin{array}{cc}
            \begin{aligned}
                \norm{\dfrac{\cos kx}{\sqrt{\pi}}}^2 &= \int_{-\pi}^{\pi}(\dfrac{\cos kx}{\sqrt{\pi}})^2dx
                =\dfrac{1}{\pi} \int_{-\pi}^{\pi} \cos^2 kx \\
                &=\dfrac{1}{\pi} \int_{-\pi}^{\pi} \dfrac{1+\cos 2kx}{2} \\
                &=\dfrac{1}{2\pi}(\dfrac{\sin 2k(\pi-\pi)}{2k}+2\pi)=1
            \end{aligned}
            &
            \begin{aligned}
                \norm{\dfrac{\sin kx}{\sqrt{\pi}}}^2 &= \int_{-\pi}^{\pi}(\dfrac{\sin kx}{\sqrt{\pi}})^2dx
                =\dfrac{1}{\pi} \int_{-\pi}^{\pi} \sin^2 kx \\
                &=\dfrac{1}{\pi} \int_{-\pi}^{\pi} \dfrac{1-\cos 2kx}{2} \\
                &=\dfrac{1}{2\pi}(\dfrac{\sin 2k(\pi-\pi)}{2k}+2\pi)=1
            \end{aligned}
        \end{array}
    \end{equation*}
随后证明这些函数两两正交.
    \begin{align*}
        \ip{\dfrac{\cos k_1x}{\sqrt{\pi}}}{\dfrac{\cos k_2x}{\sqrt{\pi}}}&
        =\dfrac{1}{2\pi} \int_{-\pi}^{\pi}(\cos((k_1+k_2)x)+\cos((k_1-k_2)x))dx=0 \\
        \ip{\dfrac{\sin k_1x}{\sqrt{\pi}}}{\dfrac{\sin k_2x}{\sqrt{\pi}}}&
        =\dfrac{1}{2\pi} \int_{-\pi}^{\pi}(-\cos((k_1+k_2)x)+\cos((k_1-k_2)x))dx=0 \\
        \ip{\dfrac{\cos k_1x}{\sqrt{\pi}}}{\dfrac{\sin k_2x}{\sqrt{\pi}}}&
        =\dfrac{1}{2\pi} \int_{-\pi}^{\pi}(-\sin((k_1+k_2)x)+\sin((k_1-k_2)x))dx=0 \\
        \ip{\dfrac{\sin k_1x}{\sqrt{\pi}}}{\dfrac{\cos k_2x}{\sqrt{\pi}}}&
        =\dfrac{1}{2\pi} \int_{-\pi}^{\pi}(\sin((k_1+k_2)x)+\sin((k_1-k_2)x))dx=0 \\
    \end{align*}
因此它们确实是两两正交且范数为\(1\)的函数,即构成一组规范正交基.

\newpage

\textit{8.}
给出\(q \in P_2(R)\)满足\(\forall p \in P_2(R),\int_{0}^{1} p(x)q(x)dx=\int_{0}^{1} p(x)\cos(\pi x)dx\).

\textit{Proof}
定义\(P_2(R)[0,1]\)上的内积为\(\ip{p}{q}=\int_{0}^{1} p(x)q(x)dx\).

定义\(\varphi \in L(V,F)\)为\(\varphi(p)=\int_{0}^{1} p(x)\cos(\pi x)dx\),验证\(\varphi\)是线性泛函.
    \begin{align*}
        \varphi(\lambda p_1+\mu p_2)&=\int_{0}^{1} (\lambda p_1+\mu p_2)(x)\cos(\pi x)dx
        =\int_{0}^{1} [\lambda p_1(x)+\mu p_2(x)]\cos(\pi x)dx \\
        &=\lambda \int_{0}^{1}p_1(x)\cos(\pi x)dx+\mu \int_{0}^{1}p_2(x)\cos(\pi x)dx
        =\lambda \varphi(p_1)+\mu \varphi(p_2)
    \end{align*}
接下来令\(\varphi(p)=\ip{p}{q}\).根据定理6.42,可以找到这样的\(q\).

根据\textit{5.B.5},\(P_2(R)\)的一组正交基是\(1,x-\dfrac{1}{2},x^2-x+\dfrac{1}{6}\).

从而\(q(x)=\varphi(1)+\varphi(x-\dfrac{1}{2})(x-\dfrac{1}{2})+\varphi(x^2-x+\dfrac{1}{6})(x^2-x+\dfrac{1}{6})\)
\(=-\dfrac{2}{\pi^2}x+\dfrac{1}{\pi^2}\).

\hspace*{\fill}

\textit{12.}
设\(\ip{\cdot}{\cdot}_1,\ip{\cdot}{\cdot}_2\)是\(V\)上的两个内积定义,其正交条件相同.

求证:存在正常数\(c\),使得\(\forall v,w \in V,\ip{v}{w}_1=c\ip{v}{w}_2\).

\textit{Proof}
给出\(\varphi_1,\varphi_2 \in L(V,F),w_0 \in V\)满足\(\forall v \in V,\varphi_1(v)=\ip{v}{w_0},\varphi_2(v)=\ip{v}{w_0}\).

由于两种内积的正交条件相同,因此\(\Ker \varphi_1=\Ker \varphi_2\).

根据\textit{3.B.30},存在常数\(c\),使得\(\varphi_1=c\varphi_2\),取\(v=w_0\)可得\(c>0\).

下证\(c\)不依赖于\(w\)的选取.考虑\(\ip{v}{w_1}_1=c_1\ip{v}{w_1}_2,\ip{v}{w_2}_1=c_2\ip{v}{w_2}_2\),要证\(c_1=c_2\).

由\(v\)的任意性,第一处\(v\)取\(w_2\),第二处\(v\)取\(w_1\),得到
    \begin{align*}
        c_2\ip{w_1}{w_2}_2=\ip{w_1}{w_2}_1=\ol{\ip{w_2}{w_1}_1}
        =\ol{c_1\ip{w_2}{w_1}_2}=\ol{c_1}\ip{w_1}{w_2}_2
    \end{align*}
因此\(c_1=\ol{c_2}\).由前面\(c>0\),得\(c_1=c_2\),因此任意的\(w\)均对应相同的\(c\).

\hspace*{\fill}

\textit{14.}
设\(e_1,\cdots,e_n\)为\(V\)的一组规范正交基.
设\(v_1,\cdots,v_n\)满足\(\forall i=1,\cdots,n,\norm{e_i-v_i}<\dfrac{1}{\sqrt{n}}\).

求证:\(v_1,\cdots,v_n\)是\(V\)的一组基.

\textit{Proof}
令\(a_1,\cdots,a_n \in F\)满足\(\sum_{i=1}^n a_iv_i=0\),下证\(a_1=\cdots=a_n=0\).
考虑\(\norm{a_i(e_i-v_i)}^2\).
    \begin{align*}
        \sum_{i=1}^n \norm{a_i(e_i-v_i)}^2<\dfrac{1}{n}\sum_{i=1}^n \abs{a_i}^2,
        n\sum_{i=1}^n \norm{a_i(e_i-v_i)}^2<\sum_{i=1}^n \abs{a_i}^2
    \end{align*}
根据\textit{6.A.12},\(n\sum_{i=1}^n \norm{a_i(e_i-v_i)}^2 \geq (\sum_{i=1}^n \norm{a_i(e_i-v_i)})^2\)
\(=(\sum_{i=1}^n \norm{a_ie_i})^2=\sum_{i=1}^n \abs{a_i}^2\).

若\(\exists i=1,\cdots,n,a_i \ne 0\),则出现矛盾,因此\(a_1=\cdots=a_n=0\).

因此\(v_1,\cdots,v_n\)线性无关,且具有正确的维数,因此是\(V\)的一组基.

\newpage

\textit{17.}
对于\(u \in V\),定义\(V\)上的线性泛函\(\varphi_u\)为\(\varphi_u(v)=\ip{v}{u},\varphi(u)=\varphi_u\).

(\textit{a})求证:若\(V\)是在\(R\)上的向量空间,则\(\varphi \in L(V,V')\).

(\textit{b})求证:若\(V\)是在\(C\)上的非零向量空间,则\(\varphi\)不是线性变换.

(\textit{c})求证:\(\varphi\)是单射变换.

(\textit{d})求证:若\(V\)是在\(R\)上的有限维向量空间,则\(\varphi\)是双射变换.

\textit{a.Proof}
    \begin{align*}
        \forall v \in V,(\varphi(\lambda u_1+\mu u_2))(v)
        &=\varphi_{\lambda u_1+\mu u_2}(v)=\ip{v}{\lambda u_1+\mu u_2} \\
        &=\lambda \ip{v}{u_1}+\mu \ip{v}{u_2}=\lambda (\varphi(u_1))(v)+\mu (\varphi(u_2))(v)
    \end{align*}
\textit{b.Proof}
    \begin{align*}
        \forall v \in V,(\varphi(\lambda u))(v)=\ol{\lambda} \ip{v}{u_1}
        =\ol{\lambda} (\varphi(u))(v) \ne \lambda (\varphi(u))(v)
    \end{align*}
\textit{c.Proof}
令\(\varphi_u=0\),则\(\forall v \in V,\varphi_u(v)=\ip{v}{u}=0\).取\(v=u\),则\(\ip{u}{u}=0,u=0\).

因此\(\Ker \varphi=\{0\}\),即\(\varphi\)是单射变换.

\textit{d.Proof}
设\(e_1,\cdots,e_n\)是\(V\)的一组规范正交基,\(\psi_1,\cdots,\psi_n\)是其对偶基.

下证\(\forall i=1,\cdots,n,\varphi(e_i)=\psi_i\).考虑任意的\(v=\sum_{i=1}^n \ip{v}{e_i}e_i\),有
    \begin{align*}
        \psi_i(\sum_{i=1}^n \ip{v}{e_i}e_i)=\ip{v}{e_i}=(\varphi(e_i))(\sum_{i=1}^n \ip{v}{e_i}e_i)
    \end{align*}
因此\(\mydim \Img \varphi=n=\mydim V=\mydim V'\),即\(\varphi\)为满射,证毕.

\hspace*{\fill}

\textit{7.A.20}
在\textit{6.B.17.d}的情况下设\(T \in L(V,W)\).求证:\(T' \circ \varphi_W=\varphi_V \circ T^*\).

\textit{Proof}
设\(\forall v \in V,w \in W\),有\((T' \circ \varphi_W)w=\varphi_w \circ T,(\varphi_V \circ T^*)w=\varphi_{T^*w}\).
    \begin{align*}
        ((T' \circ \varphi_W)w)v=(\varphi_w \circ T)v=\ip{Tv}{w}
        =\ip{v}{T^*w}=\varphi_{T^*w}v=((\varphi_v \circ T^*)w)v
    \end{align*}
\textit{6.B.17}和\textit{7.A.20}将对偶变换\(T'\)和伴随变换\(T^*\)通过\textit{Risez}表示定理结合起来.

\hspace*{\fill}

有算子范数和奇异值分解背景的\textit{16}题将单独列出.

