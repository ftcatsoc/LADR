\textit{1.}
设\(v_1,\cdots,v_m \in V\).证明:\(\{v_1,\cdots,v_m\}^\bot=\myspan(v_1,\cdots,v_m)^\bot\).

\textit{Proof}
由\(\{v_1,\cdots,v_m\} \subseteq \myspan(v_1,\cdots,v_m)\),
故\(\myspan(v_1,\cdots,v_m)^\bot \subseteq \{v_1,\cdots,v_m\}^\bot\).

下证\(\{v_1,\cdots,v_m\}^\bot \subseteq \myspan(v_1,\cdots,v_m)^\bot\),
考虑\(v \in \{v_1,\cdots,v_m\}^\bot\).

\(\forall i=1,\cdots,m,\ip{v_i}{v}=0\),则\(\ip{\sum_{i=1}^m a_iv_i}{v}=\sum_{i=1}^m a_i\ip{v_i}{v}=0\).

而\(\forall a_1,\cdots,a_m \in F,\sum_{i=1}^m a_iv_i \in \myspan(v_1,\cdots,v_m)\),
从而\(v \in \myspan(v_1,\cdots,v_m)^\bot\).

即\(\forall v \in \{v_1,\cdots,v_m\}^\bot,v \in \myspan(v_1,\cdots,v_m)^\bot\),证毕.

\hspace*{\fill}

\textit{3.}
设\(u_1,\cdots,u_m,w_1,\cdots,w_n\)是\(V\)的一组基,且\(U=\myspan(u_1,\cdots,u_m)\).

对其应用\textit{Gram-Schmidt}{\kaishu 正交化},得到\(e_1,\cdots,e_m,f_1,\cdots,f_n\).

证明:\(f_1,\cdots,f_n\)是\(U^\bot\)的一组规范正交基.

\textit{Proof}
考虑\(\forall a_i,b_i \in F,\sum_{i=1}^m a_ie_i \in \myspan(e_1,\cdots,e_m)=U,\)
\(\sum_{i=1}^n b_if_i \in \myspan(f_1,\cdots,f_n)\).

\(\ip{\sum_{i=1}^m a_ie_i}{\sum_{i=1}^n b_if_i}=\sum_{i=1}^m \sum_{i=1}^n a_i\overline{b_i} \ip{e_i}{f_i}=0\),
因此\(\forall f \in \myspan(f_1,\cdots,f_n), f\in U^\bot\).

得到\(\myspan(f_1,\cdots,f_n) \subseteq U^\bot\),结合\(\mydim U^\bot=n=\mydim \myspan(f_1,\cdots,f_n)\),

有\(U^\bot=\myspan(f_1,\cdots,f_n)\).

\hspace*{\fill}

\textit{5.}
设\(V\)是有限维内积空间且\(U\)是\(V\)的一个子空间.证明:\(P_U+P_{U^\bot}=I\).

\textit{Proof}
由于\(V=U \oplus U^\bot\),故\(\forall v \in V,v=u+u',u \in U,u' \in U^\bot\).

\(\forall v \in V,(P_U+P_{U^\bot})(v)=P_U(v)+P_{U^\bot}(v)=u+u'=v=Iv\).

\hspace*{\fill}

\textit{8.}
设\(V\)是有限维内积空间且\(P \in L(V)\)满足\(P^2=P,\forall v \in V,\norm{Pv} \leq \norm{v}\).

求证:存在\(V\)的一个子空间\(U\),使得\(P=P_U\).

\textit{Proof}
根据\textit{5.B.4},\(V= \Ker P \oplus \Img P\).猜想\(U=\Img P\).

令\(\forall v \in V,P_U v=u\),则\(P_U^2 v=P_U u=u\),
且\(\norm{P_U v}=\norm{(P_U v-v)+v} \leq \norm{P_U v-v}+\norm{v}\).

因此\(P_{\Img P}\)满足\(P\)要求的一切性质.

\hspace*{\fill}

\textit{9.}
设\(V\)是内积空间且\(T \in L(V)\),\(U\)是\(V\)的有限维子空间.

证明:\(U\)是\(T\)下的不变子空间和\(P_U T P_U=T P_U\)等价.

\textit{Proof}
充分性:\(\forall v \in V,v=u+u',u \in U,u' \in U^\bot\),且\(\forall u \in U,Tu \in U\).

\(\forall v \in V,(P_U T P_U)v=(P_U T)u=Tu=T(P_U v)=(T P_U)v\).

必要性:\(\forall u \in U,P_U(Tu)=(P_U T P_U)u=T(P_U u)=Tu\).

根据正交投影的性质,\(Tu \in U\),即\(U\)是\(T\)下的不变子空间.

\newpage

\textit{10.}
设\(V\)是有限维内积空间且\(T \in L(V)\),\(U\)是\(V\)的一个子空间.

求证:\(U\)和\(U^\bot\)都是\(T\)下的不变子空间和\(P_U T=T P_U\)等价.

\textit{Proof}
充分性:\(\forall v \in V,v=u+u',u \in U,u' \in U^\bot\),
且\(\forall u \in U,u' \in U^\bot,Tu \in U,Tu' \in U^\bot\).

\(\forall v \in V,T(P_U v)=Tu=P_U(Tu)=P_U(Tu+Tu')=(P_U T)v\).

必要性:\(\forall u \in U,P_U(Tu)=T(P_U u)=Tu\),故\(Tu \in U\),\(U\)是\(T\)的不变子空间;

\(\forall u' \in U,P_U(Tu')=T(P_U u')=0\),故\(Tu' \in U^\bot\),\(U^\bot\)是\(T\)的不变子空间.

\hspace*{\fill}

\textit{11.}
在\(R^4\)中,令\(U=\myspan((1,1,0,0),(1,1,1,2)),u \in U\),给出能使\(\norm{(1,2,3,4)-u}\)最小的\(u\).

\textit{Proof}
定义\(R^4\)上的内积为\(\ip{(x_1,x_2,x_3,x_4)}{(y_1,y_2,y_3,y_4)}=\sum_{i=1}^4 x_iy_i\).

应用正交化,易得\(U\)的一组规范正交基是\(\dfrac{1}{\sqrt{2}}(1,1,0,0),\dfrac{1}{\sqrt{5}}(0,0,1,2)\).

根据定理6.56和正交投影的性质\textit{i},有
    \begin{align*}
        u=P_U(1,2,3,4)&=\ip{(1,2,3,4)}{\dfrac{1}{\sqrt{2}}(1,1,0,0)}(\dfrac{1}{\sqrt{2}}(1,1,0,0)) \\
        &+\ip{(1,2,3,4)}{\dfrac{1}{\sqrt{5}}(0,0,1,2)}(\dfrac{1}{\sqrt{5}}(0,0,1,2))
        =(\dfrac{3}{2},\dfrac{3}{2},\dfrac{11}{5},\dfrac{22}{5})
    \end{align*}

