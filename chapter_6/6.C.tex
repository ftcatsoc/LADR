\textit{1.}
设\(v_1,\cdots,v_m \in V\).证明:\(\{v_1,\cdots,v_m\}^\bot=\myspan(v_1,\cdots,v_m)^\bot\).

\textit{Proof}
由\(\{v_1,\cdots,v_m\} \subseteq \myspan(v_1,\cdots,v_m)\),
故\(\myspan(v_1,\cdots,v_m)^\bot \subseteq \{v_1,\cdots,v_m\}^\bot\).

下证\(\{v_1,\cdots,v_m\}^\bot \subseteq \myspan(v_1,\cdots,v_m)^\bot\),
考虑\(v \in \{v_1,\cdots,v_m\}^\bot\).

\(\forall i=1,\cdots,m,\ip{v_i}{v}=0\),则\(\ip{\sum_{i=1}^m a_iv_i}{v}=\sum_{i=1}^m a_i\ip{v_i}{v}=0\).

而\(\forall a_1,\cdots,a_m \in F,\sum_{i=1}^m a_iv_i \in \myspan(v_1,\cdots,v_m)\),
从而\(v \in \myspan(v_1,\cdots,v_m)^\bot\).

即\(\forall v \in \{v_1,\cdots,v_m\}^\bot,v \in \myspan(v_1,\cdots,v_m)^\bot\),证毕.

\hspace*{\fill}

\textit{3.}
设\(u_1,\cdots,u_m,w_1,\cdots,w_n\)是\(V\)的一组基,且\(U=\myspan(u_1,\cdots,u_m)\).

对其应用\textit{Gram-Schmidt}{\kaishu 正交化},得到\(e_1,\cdots,e_m,f_1,\cdots,f_n\).

证明:\(f_1,\cdots,f_n\)是\(U^\bot\)的一组规范正交基.

\textit{Proof}
考虑\(\forall a_i,b_i \in F,\sum_{i=1}^m a_ie_i \in \myspan(e_1,\cdots,e_m)=U,\)
\(\sum_{i=1}^n b_if_i \in \myspan(f_1,\cdots,f_n)\).

\(\ip{\sum_{i=1}^m a_ie_i}{\sum_{i=1}^n b_if_i}=\sum_{i=1}^m \sum_{i=1}^n a_i\overline{b_i} \ip{e_i}{f_i}=0\),
因此\(\forall f \in \myspan(f_1,\cdots,f_n), f\in U^\bot\).

得到\(\myspan(f_1,\cdots,f_n) \subseteq U^\bot\),结合\(\mydim U^\bot=n=\mydim \myspan(f_1,\cdots,f_n)\),

有\(U^\bot=\myspan(f_1,\cdots,f_n)\).

\hspace*{\fill}

\textit{5.}
设\(V\)是有限维向量空间且\(U\)是\(V\)的一个子空间.证明:\(P_U+P_{U^\bot}=I\).

\textit{Proof}
由于\(V=U \oplus U^\bot\),故\(\forall v \in V,v=u+u',u \in U,u' \in U^\bot\).

\(\forall v \in V,(P_U+P_{U^\bot})(v)=P_U(v)+P_{U^\bot}(v)=u+u'=v=Iv\).

\hspace*{\fill}



