\textit{Theorem 5.27.}
对于任意的有限维复向量空间$V$和任意的$T \in L(V)$,总存在$V$的一组基,

使得线性算子$T$的矩阵是上三角矩阵.

\textit{Proof}:
使用数学归纳法.$\mydim V=1$的情况显然成立.

现在假设对于任意的$W$满足$\mydim W<\mydim V$,$W$都满足条件.

根据定理5.21,该复向量空间必存在一个特征值$\lambda$及对应的特征向量$v_1$.

令$U=\myspan(v_1)$,考虑商空间$V/U$,$\mydim V/U=n-1$,因此可以对其使用假设.

构造$T$在$V/U$上的诱导变换$T/U$满足$\forall v+U \in V/U,(T/U)(v+U)=Tv+U$.

根据假设,$V/U$存在一组基$v_2+U,\cdots,v_n+U$满足$(T/U)(v_i+U) \in \myspan(v_2+U,\cdots,v_n+U)$.

根据\textit{3.E.13},$v_1,\cdots,v_n$是$V$的一组基,
且满足$\forall i=1,\cdots,n,Tv_i \in \myspan(v_1,\cdots,v_i)$.

\hspace*{\fill}

\textit{1.}
设$T \in L(V)$且存在$n \in N^+$使得$T^n=0$.
求证:$I-T$是可逆,且$(I-T)^{-1}=\sum_{i=0}^{n-1} T^i$.

\textit{Proof}:
$(I-T)\sum_{i=0}^{n-1} T^i=\sum_{i=0}^{n-1} T^i-\sum_{i=1}^n T^i=I-T^n=I$.

\hspace*{\fill}

\textit{3.}
设$T \in L(V)$,满足$T^2=I$且$-1$不是$T$的特征值.证明$T=I$.

\textit{Proof}:
$T^2=I \Rightarrow (T+I)(T-I)=0$,因此$1$和$-1$至少有一个是$T$的特征值.

而$-1$不是$T$的特征值,因此$T=I$.

\hspace*{\fill}

\textit{4.}
设$P \in L(V)$满足$P^2=P$.证明$V=\Ker P \oplus \Img P$.

\textit{Proof}:
对于$\forall v \in V,v=Pv+(v-Pv)$.

显然$Pv \in \Img P$,且$P(v-Pv)=(P-P^2)v=0 \Rightarrow v-Pv \in \Ker P$,因此$V=\Img P+\Ker P$.

再考虑$v \in \Ker P \cap \Img P$,因此$\exists u \in V$,使得$Pu=v$,得$v=Pu=P^2u=Pv=0$,证毕.

\hspace*{\fill}

\textit{5.}
设$S,T \in L(V)$且$S$是可逆变换.设$p \in P(F)$是一个多项式.

求证:$p(STS^{-1})=Sp(T)S^{-1}$

\textit{Proof}:
对于$Sv \in V$,有$(STS^{-1})^m (Sv)=(STS^{-1})^{m-1}(S(Tv))=\cdots=(ST^m)v$.
    \begin{align*}
        p(STS^{-1})(Sv)=\sum_{i=0}^m a_i(STS^{-1})^m (Sv) 
        =\sum_{i=0}^m a_i(ST^m)v=Sp(T)v=Sp(T)S^{-1}(Sv)
    \end{align*}

\newpage

\textit{6.}
设$T \in L(V)$且$U$是$T$下的不变子空间.

求证:对于任意多项式$p \in P(F)$,$U$都是$p(T)$下的不变子空间.

\textit{Proof}:
使用数学归纳法.先验证$n=1$的情况.

$\forall u \in U,(a_0I+a_1T)u=a_0u+a_1(Tu) \in U$,情况成立.

接着假设结论在次数为$n$时成立,即$\forall u \in U,p_n(T)u=\sum_{i=0}^n a_iT^iu \in U$.

下面验证次数为$n+1$的情况.
$\forall u \in U,p_{n+1}(T)u=\sum_{i=0}^{n+1} a_iT^iu=\sum_{i=0}^n a_iT^iu+a_{n+1}T^{n+1}u$.

由于$\sum_{i=0}^n a_iT^iu \in U,a_{n+1}T^{n+1}u \in U$,故$p_{n+1}(T)u \in U$,证毕.

\hspace*{\fill}

\textit{9.}
设$V$是有限维向量空间且$T \in L(V)$,存在$v \ne 0 \in V$.

令$p$是能使得$p(T)v=0$的次数最低的多项式.求证:$p$的所有零点都是$T$的特征值.

\textit{Proof}:
使用反证法.设$\exists \lambda \in F,p(\lambda)=0$但$T-\lambda I$可逆.

根据代数基本定理,存在$q(T) \in L(V)$,使得$p(T)=(T-\lambda I)q(T),\mydeg q<\mydeg p$.

因此$q(T)$不能满足$q(T)v=0$.但是$v \ne 0$,因而$(T-\lambda I)v \ne 0$.

因此$p(T)v=(T-\lambda I)q(T)v \ne 0$,矛盾,从而原命题得证.

\hspace*{\fill}

\textit{11.}
设$T \in L(V)$和一个多项式$p \in P(C)$,$\alpha \in C$.

求证:$\alpha$是$p(T)$的特征值和存在$T$的特征值$\lambda$使得$\alpha=p(\lambda)$等价.

\textit{Proof}:
必要性:若$Tv=\lambda v$且$\alpha=p(\lambda)$,
则$p(T)v=\sum_{i=0}^n a_iT^iv=\sum_{i=0}^n a_i\lambda^iv=p(\lambda)v$.

充分性:$\exists v \ne 0,Tv=\alpha v \Rightarrow (p(T)-\alpha I)$不可逆.

因此$p(T)-\alpha I=c\prod_{i=1}^n (T-\lambda_i I)$不可逆,即至少存在一个$\lambda_i$使得$T-\lambda_i I$不可逆,

即$\lambda_i$是$T$的特征值,因此$p(T)v=p(\lambda_i)v=\alpha v$.

\hspace*{\fill}

\textit{13.}
设$W$是复向量空间且$T \in L(W)$没有特征值.

求证:若$U$在$T$下不变,则$U=\{0\}$或者$U$为无限维向量空间.

\textit{Proof}:
$\{0\}$的情况显然成立.现在假设$U$是非零有限维复向量空间.

则$T|_U \in L(U)$必有特征值$\lambda$,即$T|_U u=\lambda u$.
从而$T \in L(W)$实际上有一个特征值$\lambda$,矛盾.

\hspace*{\fill}

\textit{16.}
令$\varphi:P_n(C) \rightarrow V$为$\varphi p=p(T)v$,以此证明定理5.21.

\textit{Proof}:
先证明$\varphi$是线性变换.
    \begin{align*}
        \varphi(\lambda p+\mu q)=(\lambda p+\mu q)(T)v=(\lambda p(T)+\mu q(T))v
        =\lambda p(T)v+\mu q(T)v=\lambda(\varphi p)+\mu(\varphi q)
    \end{align*}
由于$\mydim P_n(C)=n+1,\mydim V=n$,因此该变换不是单射变换,

也即存在$v \ne 0$使得$p(T)v=\sum_{i=0}^n a_i(T^iv)=0$,后文证明与原文相同.

