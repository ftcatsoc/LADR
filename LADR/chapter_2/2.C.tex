\textit{10.}
设$p_0,p_1,\cdots,p_m\in P(F)$,其中$p_j$是次数为$j$的多项式.

求证:$p_0,p_1,\cdots,p_m$是$P_m(F)$的一组基.

\textit{Proof}:
使用数学归纳法.先验证$m=0$的情况,显然成立.

随后假设$j=m$时结论成立,需证$\myspan (p_0,p_1,\cdots,p_m,p_{m+1})=\myspan (1,x,\cdots,x^m,x^{m+1})$.

下证$1^{\circ}\myspan (p_0,p_1,\cdots,p_m,p_{m+1}) \subseteq \myspan (1,x,\cdots,x^m,x^{m+1})$

并且$2^{\circ}\myspan (1,x,\cdots,x^m,x^{m+1}) \subseteq \myspan (p_0,p_1,\cdots,p_m,p_{m+1})$.

$1^{\circ}$的证明是显然的.$\forall p\in \myspan (p_0,p_1,\cdots,p_m,p_{m+1}),p\in P_{m+1}(F)$.

即$\myspan (p_0,p_1,\cdots,p_m,p_{m+1})\subseteq \myspan (1,x,\cdots,x^m,x^{m+1})$.

下证$2^{\circ}$成立.设选定的$p_{m+1}=a_{m+1}x^{m+1}+\sum_{i=0}^m a_ix^i(a_{m+1} \ne 0)$.

变形得到$x^{i+1}=(p_{m+1}-\sum_{i=0}^m a_ix^i)/a_{m+1} \in \myspan (1,x,\cdots,x^m,p_{m+1})$.

而$\myspan (p_0,p_1,\cdots,p_m)=\myspan (1,x,\cdots,x^m)$,

故$\myspan (1,x,\cdots,x^m,p_{m+1})=\myspan (p_0,p_1,\cdots,p_m,p_{m+1})$.

从而$x^{m+1}\in \myspan (p_0,p_1,\cdots,p_m,p_{m+1})$,

进而$\myspan (1,x,\cdots,x^m,x^{m+1}) \subseteq \myspan (p_0,p_1,\cdots,p_m,p_{m+1})$.

如此$1^{\circ}$和$2^{\circ}$均成立且向量均线性无关,即该张成组确实是$P_m(F)$的一组基.

\hspace*{\fill}

\textit{14.}
设$U_1,\cdots,U_m$都是$V$的有限维子空间,且$U_1,\cdots,U_m$相互独立.

求证:$\sum_{i=1}^m U_i$是有限维向量空间,且$\mydim \sum_{i=1}^m U_i=\sum_{i=1}^m \mydim U_i$.

\textit{Proof}:
设$U_i$的一组基为$u_1^i,\cdots,u_{a_i}^i$,
则$\sum_{i=1}^m U_i=\myspan(u_1^1,\cdots,u_{a_1}^1,\cdots,u_{a_m}^1,\cdots,u_{a_m}^m)$.

因而$\mydim \sum_{i=1}^m U_i \leq \sum_{i=1}^m \mydim U_i$,即$\sum_{i=1}^m U_i$是有限维向量空间.

下面使用数学归纳法,先验证$U_1+U_2$的情况.

由于$\mydim (U_1+U_2)=\mydim U_1+\mydim U_2-\mydim (U_1\cap U_2)$且$U_1\cap U_2=\{0\}$,

故$\mydim (U_1+U_2)=\mydim U_1+\mydim U_2$,从而$n=2$的情况成立.

随后假设$n=m$时结论成立,则当$n=m+1$时,

有$\mydim (\sum_{i=1}^m U_i+U_{m+1})=\mydim \sum_{i=1}^m U_i+\mydim U_{m+1}-\mydim (\sum_{i=1}^m U_i \cap U_{m+1})$.

由于$n=m$时结论成立,故$\mydim \sum_{i=1}^m U_i=\sum_{i=1}^m \mydim U_i$.

又由于$U_i$相互独立,故$\sum_{i=1}^m U_i \cap U_{n+1}=\{0\}$.
故$\mydim \sum_{i=1}^m U_i=\sum_{i=1}^m \mydim U_i$,证毕.

