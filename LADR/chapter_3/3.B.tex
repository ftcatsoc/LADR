\textit{3.}
设$v_1,\cdots,v_m \in V$,定义$T \in L(F^m,V)$为$T(z_1,\cdots,z_m)=\sum_{i=1}^m z_iv_i$.

(\textit{a})若$V=\myspan (v_1,\cdots,v_m)$,则$T$具有怎样的性质?

(\textit{b})若$v_1,\cdots,v_m$线性无关,则$T$具有怎样的性质?

\textit{a.Proof}:
由于$\Img T=\myspan (v_1,\cdots,v_m)=V$,故$T$满射.

\textit{b.Proof}:
令$T(z_1,\cdots,z_m)=\sum_{i=1}^m z_iv_i=0$.

由于$v_1,\cdots,v_m$线性无关,故$z_1=\cdots=z_m=0$,即$\Ker T=\{0\}$,从而$T$单射.

\hspace*{\fill}

\textit{9.}
设$T\in L(V,W)$是单射的,且$v_1,\cdots,v_m$是$V$中一列线性无关的向量.

求证:$Tv_1,\cdots,Tv_m$在$W$中线性无关.

\textit{Proof}:
令$\sum_{i=1}^m a_iTv_i=0$,$Tv_1,\cdots,Tv_m$在$W$中线性无关即要证$a_1=\cdots=a_m=0$.

对于$\sum_{i=1}^m a_iTv_i=T\sum_{i=1}^m a_iv_i=0$,由于$T$是单射变换,故$\sum_{i=1}^m a_iv_i=0$.

而$v_1,\cdots,v_m$线性无关,则$a_1=\cdots=a_m=0$,证毕.

\hspace*{\fill}

\textit{10.}
设$V=\myspan (v_1,\cdots,v_m)$且$T\in L(V,W)$,求证:$\Img T=\myspan (Tv_1,\cdots,Tv_m)$.

\textit{Proof}:
$\Img T=\{Tv|v\in V\}=\{Tv|v\in \myspan(v_1,\cdots,v_m)\}$.

$\forall u=T\sum_{i=1}^m a_iv_i \in \Img T,u=\sum_{i=1}^m a_iTv_i \in \myspan(Tv_1,\cdots,Tv_m)$;

$\forall u=\sum_{i=1}^m a_iTv_i \in \myspan(Tv_1,\cdots,Tv_m),u=T\sum_{i=1}^m a_iv_i \in \Img T$.

因而有$\Img T \subseteq \myspan(Tv_1,\cdots,Tv_m),\myspan(Tv_1,\cdots,Tv_m) \subseteq \Img T$.

即$\Img T=\myspan (Tv_1,\cdots,Tv_m)$,证毕.

\hspace*{\fill}

\textit{12.}
设$V$是有限维向量空间且$T\in L(V,W)$.

求证:存在$V$的一个子空间$U$,满足$U \cap \Ker T=\{0\}$且$\Img T=\{Tu|u\in U\}$.

\textit{Proof}:
令$v_1,\cdots,v_m$为$\Ker T$的一组基,故存在线性无关的$u_1,\cdots,u_n$,

使得$v_1,\cdots,v_m,u_1,\cdots,u_n$为$V$的一组基.

令$U=\myspan (u_1,\cdots,u_n)$,则$U\cap \Ker T=\{0\}$显然成立,下证$\Img T=\{Tu|u\in U\}$.

令$\forall w\in V,\exists u\in U,v\in \Ker T, w=u+v$,从而
    \begin{align*}
        \Img T=\{Tw|w\in V\}=\{T(u+v)|u\in U,v\in \Ker T\}=\{Tu|u\in U\}
    \end{align*}
从而$\Img T=\{Tu|u\in U\}$,证毕.

\newpage

\textit{17.}
设$V$和$W$是有限维的.求证:当且仅当$\mydim V\leq \mydim W$,存在单射的$T\in L(V,W)$.

\textit{Proof}:
必要性:若存在单射的$T\in L(V,W)$,

则$\mydim V=\mydim \Ker T+\mydim \Img T=\mydim \Img T \leq \mydim W$,证毕.

充分性:设$v_1,\cdots,v_m$和$w_1,\cdots,w_n$分别为$V$和$W$的一组基.

有$\mydim V=m\leq n=\mydim W$,从而定义$Tv_i=w_i,i=1,\cdots,m$.

定理3.5保证了线性变换$T$的存在性,下证$T$是单射的.

令$T\sum_{i=1}^m a_iv_i=\sum_{i=1}^m a_iTv_i=\sum_{i=1}^m a_iw_i=0$,从而$a_1=\cdots=a_m=0$,证毕.

\hspace*{\fill}

\textit{19.}
设$V$和$W$是有限维的,$U$是$V$的一个子空间.

求证:当且仅当$\mydim U \geq \mydim V-\mydim W$,存在$T\in L(V,W)$,满足$\Ker T=U$.

\textit{Proof}:
必要性:原式等价于$\mydim U=\mydim \Ker T=\mydim V-\mydim \Img T \geq \mydim V-\mydim W$,

即$\mydim \Img T \leq \mydim W$,证毕.

充分性:设$v_1,\cdots,v_m$是$U$的一组基.由于$U$是$V$的一个子空间,

故存在线性无关的$v_{m+1},\cdots,v_{m+n}$使得$v_1,\cdots,v_{m+n}$是$V$的一组基.

从而原式等价于$m\geq (m+n)-\mydim W$,即$\mydim W\geq n$.令
    \begin{align*}
        Tv_i=0,i=1,\cdots,m \quad Tv_i=w_i,i=m+1,\cdots,m+n
    \end{align*}
定理3.5保证了线性变换$T$的存在性.

$\Ker T=U$显然成立,同时$\Img T=\myspan (Tv_{m+1},\cdots,Tv_{m+n})$.

由于$\mydim \Img T\leq \mydim W$,这要求$n\leq \mydim W$,与条件相符.

\hspace*{\fill}

\textit{20.}
设$V$和是有限维向量空间且$T\in L(V,W)$.

求证:$T$是单射变换与存在$S\in L(W,V)$满足$ST$是在$V$上的单位变换等价.

\textit{Proof}:
必要性:利用反证法.设$T$不是单射变换,
即$\exists v_\alpha,v_\beta \in V$满足$Tv_\alpha=Tv_\beta$且$v_\alpha \ne v_\beta$.

$\exists S,ST=I$,因而有$v_\alpha=S(Tv_\alpha)=S(Tv_\beta)=v_\beta$.构成矛盾,即$T$为单射变换,证毕.

充分性:设$v_1,\cdots,v_m$是$V$的一组基.

由于$T$是单射变换,根据\textit{3.B.9},$Tv_1,\cdots,Tv_m$线性无关.

因而存在$w_1,\cdots,w_n$,使得$Tv_1,\cdots,Tv_m,w_1,\cdots,w_n$是$W$的一组基.定义
    \begin{align*}
        S(Tv_i)=v_i,i=1,\cdots,m \quad Sw_j=0,j=1,\cdots,n
    \end{align*}
定理3.5保证了线性变换$S$的存在性.下证$ST=I$.
    \begin{align*}
        \forall v=\sum_{i=1}^m a_iv_i\in V,(ST)v=S(\sum_{i=1}^m a_iTv_i)
        =\sum_{i=1}^m a_iS(Tv_i)=\sum_{i=1}^m a_iv_i=v
    \end{align*}

\newpage

\textit{21.}
设$V$是有限维向量空间且$T\in L(V,W)$.

求证:$T$是满射变换与存在$S\in L(W,V)$满足$TS$是在$W$上的单位变换等价.

\textit{Proof}:
必要性:利用反证法.设$T$不是满射变换,即$\exists w\in W,w\notin \Img T$.

$\exists S,TS=I$,从而$T(Sw)=w,w\in \Img T$,矛盾,必要性得证.

充分性:设$w_1,\cdots,w_m$是$W$的一组基.由于$T$是满射变换,$\Img T=\myspan (w_1,\cdots,w_m)$.

设$v_1,\cdots,v_m\in V,V=\myspan (v_1,\cdots,v_m,u_1,\cdots,u_n)$,则定义
    \begin{align*}
        Sw_i=v_i,i=1,\cdots,m \quad
        Tv_i=w_i,i=1,\cdots,m \quad Tu_j=0,j=1,\cdots,n
    \end{align*}
定理3.5保证了线性变换$S$的存在性.下证$TS=I$.
    \begin{align*}
        \forall w=\sum_{i=1}^m a_iw_i\in W,(TS)w=T(\sum_{i=1}^m a_iSw_i) 
        =\sum_{i=1}^m a_iTv_i=\sum_{i=1}^m a_iw_i=w
    \end{align*}

\hspace*{\fill}

\textit{24.}
设$W$是有限维向量空间且$T_1,T_2 \in L(V,W)$.

求证:$\Ker T_1 \subseteq \Ker T_2$的充要条件是$\exists S \in L(W)$,使得$T_2=ST_1$.

\textit{Proof}:
必要性:设$\forall v \in \Ker T_1$,有$T_2v=ST_1v=0$.

故$\forall v \in \Ker T_1, v \in \Ker T_2$,即$\Ker T_1 \subseteq \Ker T_2$,必要性得证.

充分性:$W$是有限维向量空间且$\Img T_1 \subseteq W$,故$\Img T_1$也是有限维向量空间.

令$T_1v_1,\cdots,T_1v_m$是$\Img T_1$的一组基且$\sum_{i=1}^m a_iv_i=0$,根据\textit{3.A.4},

有$\sum_{i=1}^m a_iT_1v_i=T_1(\sum_{i=1}^m a_iv_i)=0$,得$a_1=\cdots=a_m=0$,从而$v_1,\cdots,v_m$线性无关.

令$K=\myspan (v_1,\cdots,v_m)$,从而$V=K \oplus \Ker T$.现在定义线性变换$S$.

由于$T_1v_1,\cdots,T_1v_m$线性无关,故将其补充为$W$的一组基$T_1v_1,\cdots,T_1v_m,w_1,\cdots,w_n$.
    \begin{align*}
        S(T_1v_i)=T_2v_i,i=1,\cdots,m \quad Sw_j=0,j=1,\cdots,n
    \end{align*}
定理3.5保证了线性变换$S$的存在性.下证$ST_1=T_2$.

由于$\forall v \in V, v=v_0+\sum_{i=1}^m a_iv_i$,其中$v_0 \in \Ker T_1$,故
    \begin{align*}
        S(T_1v)=S(T_1v_0)+\sum_{i=1}^m a_iS(T_1v_i)
        =\sum_{i=1}^m a_iT_2v_i=T_2v_0+\sum_{i=1}^m a_iT_2v_i=T_2v
    \end{align*}
由$\Ker T_1 \subseteq \Ker T_2$,上式成立.因此$\forall v \in V, ST_1v=T_2v$,证毕.

\newpage

\textit{25.}
设$V$是有限维向量空间且$T_1,T_2 \in L(V,W)$.

求证:$\Img T_1 \subseteq \Img T_2$的充要条件是$\exists S \in L(V)$,使得$T_1=T_2S$.

\textit{Proof}:
必要性:设$\forall v \in V$,有$T_1v=T_2Sv \in \Img T_2$.

故$\forall v \in \Img T_1, v \in \Img T_2$,即$\Img T_1 \subseteq \Img T_2$,必要性得证. 

充分性:设$u_1,\cdots,u_m$是$V$的一组基,从而$\Img T_1=\myspan(u_1,\cdots,u_m) \subseteq \Img T_2$.

因此$\exists v_1,\cdots,v_m \in V$,使得$T_1u_i=T_2v_i,i=1,\cdots,m$.定义$S$为
    \begin{align*}
        Su_i=v_i,i=1,\cdots,m
    \end{align*}
定理3.5保证了线性变换$S$的存在性.
    \begin{align*}
        \forall u=\sum_{i=1}^m a_iu_i \in V,T_2S\sum_{i=1}^m a_iu_i=T_2\sum_{i=1}^m a_i(Su_i)
        =\sum_{i=1}^m a_iT_2v_i=\sum_{i=1}^m a_iT_1u_i=T_1\sum_{i=1}^m a_iu_i
    \end{align*}
即$\forall u \in V , T_1=T_2S$,充分性得证.

\hspace*{\fill}

    \begin{comment}
        \textit{26.}
        设$D \in L(P(R))$对于其中任意的多项式$p$满足$\mydeg Dp=\mydeg p-1$.

        求证:$D$是满射变换.

        \textit{Proof}:
        根据\textit{3.B.10},命题等价于
            \begin{align*}
                \myspan (D(x),D(x^2),\cdots)=\Img D=P(R)=\myspan (1,x,\cdots)
            \end{align*}
        根据\textit{2.C.10},由于$\mydeg Dp=\mydeg p-1$,

        故$\myspan (D(x),D(x^2),\cdots)=\myspan (1,x,\cdots)$成立.
    \end{comment}

\textit{28.}
设$T \in L(V,W)$,且$w_1,\cdots,w_m$是$\Img T$的一组基.

求证:$\exists \varphi_1,\cdots,\varphi_m \in L(V,F)$,
故$\forall v \in V$均满足$Tv=\sum_{i=1}^m{\varphi_i(v)w_i}$.

\textit{Proof}:
由于$w_1,\cdots,w_m$是$\Img T$的一组基,
故$\forall Tv \in \Img T,\exists! a_i \in F$,使得$Tv=\sum_{i=1}^m a_iw_i$.

因此定义$\varphi_i(v)=a_i$即可,下证$\forall i=1,\cdots,m,\varphi_i \in L(V,F)$.

即证$\forall i=1,\cdots,m,\forall v_\alpha,v_\beta \in V$,
$\varphi_i(\lambda v_\alpha+\mu v_\beta)=\lambda \varphi_i(v_\alpha)+\mu \varphi_i(v_\beta)$.
    \begin{align*}
        T(\lambda v_\alpha+\mu v_\beta)=\sum_{i=1}^m \lambda \varphi_i(v_\alpha)w_i+\sum_{i=1}^m \mu \varphi_i(v_\beta)w_i
        =\sum_{i=1}^m \varphi_i(\lambda v_\alpha+\mu v_\beta)w_i=\lambda Tv_\alpha+\mu Tv_\beta
    \end{align*}
综上,$\forall i=1,\cdots,m,\varphi_i \in L(V,F)$.

\hspace*{\fill}

\textit{29.}
设$\varphi \in L(V,F)$.设$u \in V \notin \Ker \varphi$.求证:$V=\Ker \varphi \oplus \{au|a \in F\}$.

\textit{Proof}:
根据\textit{3.B.12},存在$V$的一个子空间$K$,

使得$V=\Ker \varphi \oplus K$且$\Img \varphi=\{\varphi (v)|v \in K\}$,因此只需证明$K=\{au|a \in F\}$即可.

对于$\forall v \in K$,$\varphi(v) \in F$.而$\varphi(u) \ne 0 \in F$,
故$\exists a \in F$,使得$\varphi(v)=a \varphi(u)=\varphi(au),a \ne 0$.

所以$\Img \varphi=\{\varphi (au)|a \in F\}$,即$K=\{au|a \in F\}$,证毕.

\hspace*{\fill}

\textit{30.}
设$\varphi_1,\varphi_2 \in L(V,F)$且$\Ker  \varphi_1=\Ker  \varphi_2$.
求证:存在常数$c \in F$,使得$\varphi_1=c \varphi_2$.

\textit{Proof}:
若$\Ker \varphi_1=\Ker \varphi_2=V$,则$\varphi_1=\varphi_2=0$,故$c$可以为任意常数.

若$\exists u \notin \Ker \varphi,u \in V$,
则根据\textit{3.B.29},$\forall v \in V$,$v$可以被唯一分解为$v_0+a_vu$.

其中$v_0 \in \Ker \varphi,a_v \in F$.

从而$\varphi_1(v)=\varphi_1(v_0+a_vu)=a_v \varphi_1(u)$
并且$\varphi_2(v)=\varphi_2(v_0+a_vu)=a_v \varphi_2(u)$.
    \begin{align*}
        \frac{\varphi_1(v)}{\varphi_2(v)}=\frac{\varphi_1(u)}{\varphi_2(u)}=\mathrm{cons.}
    \end{align*}
由于$u \notin \Ker \varphi_2$,故$\varphi_2(u) \ne 0$,该式恒成立,证毕.

