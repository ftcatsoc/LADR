\textit{1.}
设\(T \in L(V)\)且\(U\)是\(V\)的一个子空间.

(\textit{a})证明:若\(U \subseteq \Ker T\),则\(U\)是\(T\)的不变子空间.

(\textit{b})证明:若\(\Img T \subseteq U\),则\(U\)是\(T\)的不变子空间.

\textit{a.Proof}
\(\forall u \in U,Tu=0 \in U\).

\textit{b.Proof}
\(\forall u \in U,Tu \in \Img T \subseteq U\).

\hspace*{\fill}

\textit{2.}
设\(S,T \in L(V)\)满足\(ST=TS\).

(\textit{a})证明:\(\Ker S\)是\(T\)的不变子空间.

(\textit{b})证明:\(\Img S\)是\(T\)的不变子空间.

\textit{a.Proof}
\(\forall v \in \Ker S,Sv=0 \Rightarrow STv=TSv=0 \Rightarrow Tv \in \Ker S\).

\textit{b.Proof}
\(\forall Sv \in \Img S,T(Sv)=S(Tv) \in \Img S\).

\hspace*{\fill}

\textit{4.}
设\(T \in L(V)\)且\(U_1,\cdots,U_m\)都是\(V\)的不变子空间.求证:\(\sum_{i=1}^m U_i\)是 \(V\)的不变子空间.

\textit{Proof}
由于\(\forall u \in \sum_{i=1}^m U_i,\exists u_i \in U_i,u=\sum_{i=1}^m u_i\)且\(U_1,\cdots,U_m\)都是\(V\)的不变子空间,

得到\(\forall i=1,\cdots,m,Tu_i \in U_i,Tu=\sum_{i=1}^m Tu_i \in \sum_{i=1}^m U_i\).

\hspace*{\fill}

\textit{5.}
设\(T \in L(V)\)且\(U_1,\cdots,U_m\)都是\(V\)的不变子空间.
求证:\(\bigcap_{i=1}^m U_i\)是 \(V\)的不变子空间.

\textit{Proof}
设\(u \in \bigcap_{i=1}^m U_i\),则\(\forall i=1,\cdots,m,Tu \in U_i\).
因此\(\forall u \in \bigcap_{i=1}^m U_i,Tu \in \bigcap_{i=1}^m U_i\),证毕.

\hspace*{\fill}

\textit{6.}
证明或给出反例:若\(V\)是有限维向量空间且\(V\)是\(U\)的一个子空间.

若\(U\)满足对于\(V\)上的任意算子\(T\),\(U\)都是\(T\)的不变子空间,则有\(U=\{0\}\)或\(U=V\).

\textit{Proof}
由于\(U\)是\(V\)的一个子空间且\(V\)是有限维向量空间,不妨设\(\{0\} \subset U \subset V\).

令\(u_1,\cdots,u_m\)为\(U\)的一组基,\(u_1,\cdots,u_m,v_1,\cdots,v_n\)是\(V\)的一组基.

故一定存在\(T \in L(V)\),使得\(Tu_i=v_j \notin U,i=1,\cdots,m,j=1,\cdots,n\).

从而\(U\)不是不变子空间,因此\(U\)只能为\(\{0\}\)或\(V\).

\hspace*{\fill}

\textit{8.}
定义\(T \in L(F^2)\)为\(T(w,z)=(z,w)\).给出\(T\)的所有特征值和特征向量.

\textit{Proof}
设\(T(w,z)=\lambda(w,z)\),得到\(\lambda w=z\)且\(\lambda z=w\),联立得到\(\lambda_1=1,\lambda_2=-1\).

当\(\lambda=1\)时,对应的特征向量为\((1,1)\);当\(\lambda=-1\)时,对应的特征向量为\((1,-1)\).

\newpage

\textit{10.}
定义\(T \in L(F^n)\)为\(T(x_1,\cdots,x_n)=(x_1,\cdots,nx_n)\).

(\textit{a})给出\(T\)的所有特征值和特征向量.

(\textit{b})给出\(T\)的所有不变子空间.

\textit{a.Proof}
\(T(x_1,\cdots,x_n)=(\lambda x_1,\cdots,\lambda x_n)=(x_1,\cdots,nx_n)\).
因此只能有\(\lambda_i=i,x_{j \ne i}=0\).

故\(T\)共有\(n\)个特征值\(1,\cdots,n\),第\(i\)个特征值对应的特征向量是\((0,\cdots,1^i,\cdots,0)\).

\textit{b.Proof}
设\(e_1,\cdots,e_n\)是\(F^n\)的标准基.

根据\textit{5.A.10.a},\(\forall i=1,\cdots,n,U_i=\myspan (e_i)\)都是\(T\)的不变子空间.

又根据\textit{5.A.4},其中任意若干\(U_i\)之和都是\(T\)的不变子空间.

\hspace*{\fill}

\textit{12.}
定义\(T \in P_4(R)\)为\((Tp)(x)=xp'(x)\).给出\(T\)的所有特征值和特征向量.

\textit{Proof}
\((Tp)(x)=xp'(x)=\sum_{i=1}^4 ia_ix^i=\lambda p(x)=\lambda \sum_{i=0}^m a_ix_i\).

得到\(a_0=0\)且\(\lambda a_i=i a_i\).
因此\(\lambda=1,2,3,4\),对应的特征向量为\(a_1x,a_2x^2,a_3x^3,a_4x^4\).

\hspace*{\fill}

\textit{14.}
设\(V=U \oplus W\),其中\(U\)和\(W\)都是\(V\)的非零子空间.

定义\(P \in L(V)\)为\(P(u+w)=u,u \in U,w \in W\).给出\(T\)的所有特征值和特征向量.

\textit{Proof}
\(P(u+w)=\lambda(u+w)=u \Rightarrow\) \(\lambda=1,w=0\)或\(\lambda=0,u=0\).

因此\(T\)有\(0\)和\(1\)两个特征值,对应的特征向量分别是\(W\)和\(U\)中所有非零向量.

\hspace*{\fill}

\textit{15.}
设\(T \in L(V)\)且\(S \in L(V)\)是可逆变换.

(\textit{a})证明:\(T\)和\(S^{-1}TS\)有相同的特征值.

(\textit{b})给出\(T\)和\(S^{-1}TS\)的特征向量之间的关系.

\textit{a.Proof}
设\(\lambda \in F\)和\(v \in V\)满足\(Tv=\lambda v\).
考虑\(S^{-1}v\).有\((S^{-1}TS)(S^{-1}v)=(S^{-1}T)v=\lambda S^{-1}v\).

因此\(Tv=\lambda v \Rightarrow (S^{-1}TS)(S^{-1}v)=\lambda S^{-1}v\).

\textit{b.Proof}
根据\textit{5.A.15.a},
\(S^{-1}v\)是\(S^{-1}TS\)的特征向量等价于\(v\)是\(T\)的特征向量.

\hspace*{\fill}

\textit{21.}
设\(T \in L(V)\)是可逆变换.

(\textit{a})设\(\lambda \in F\)且\(\lambda \ne 0\).
证明:\(\lambda\)是\(T\)的特征值和\(\lambda^{-1}\)是\(T^{-1}\)的特征值等价.

(\textit{b})证明\(T\)和\(T^{-1}\)有相同的特征向量.

\textit{a.Proof}
\(Tv=\lambda v \Rightarrow T^{-1}Tv=\lambda T^{-1}v \Rightarrow T^{-1}v=\lambda^{-1} v\).

\textit{b.Proof}
根据\textit{5.A.21.a},若\(v\)是\(T\)的一个特征向量,则\(v\)也是\(T^{-1}\)的一个特征向量.

\newpage

\textit{23.}
设\(V\)是有限维向量空间且\(S,T \in L(V)\).求证:\(ST\)和\(TS\)有相同的特征值.

\textit{Proof}
设\((ST)v=\lambda v\).考虑\(Tv\),有\((TS)(Tv)=T((ST)v)=T(\lambda v)=\lambda Tv\).

若\(v \notin \Ker T\),原式得证;若\(v \in \Ker T\),则\(ST\)和\(TS\)都有特征值\(0\).

\hspace*{\fill}

\textit{24.}
设\(A\)是\(n-n\)矩阵.定义\(T \in L(F^n)\)为\(Tx=Ax\).

(\textit{a})矩阵每行元素之和均为\(1\).证明\(1\)是矩阵的特征值.

(\textit{b})矩阵每列元素之和均为\(1\).证明\(1\)是矩阵的特征值.

\textit{a.Proof}
不妨先猜测特征向量.定义\(x \in F^n\)为\((1,\cdots,1)^T\).
    \begin{align*}
        (A-I)x=
            \begin{pmatrix}
                \sum_{j=1}^n A_{1,j}-1 & \cdots & \sum_{j=1}^n A_{n,j}-1
            \end{pmatrix}^T
            =0
    \end{align*}
因此\(1\)确实是\(T\)的特征值.

\textit{b.Proof}
    \begin{align*}
        x^T(A-I)=
            \begin{pmatrix}
                \sum_{i=1}^n A_{i,1}-1 & \cdots & \sum_{i=1}^n A_{i,n}-1
            \end{pmatrix}
                =0
    \end{align*}
因此\(1\)确实是\(T\)的特征值.

\hspace*{\fill}

\textit{28.}
设\(V\)是有限维向量空间.

\(T \in L(V)\)满足任意维数为\(\mydim V-1\)的子空间都是不变子空间.求证:\(T=\lambda I,\lambda \in F\).

\textit{Proof}
设\(v_1,\cdots,v_m\)是\(V\)的一组基.则\(\exists a_i^j \in F,Tv_j=\sum_{i=1}^m a_i^j v_i\).

我们不妨先考虑所有包含\(\myspan (v_1)\)的且维数为\(\mydim V-1\)的子空间.

设\(U_i=\myspan (v_i)\),这些子空间的统一形式可以被写作\(\sum_{i=1}^m U_i(i \ne k,k \ne 1)\).

由于该子空间不变,故\(Tv_1=\sum_{i=1}^m a_i^1 v_i \in \sum_{i=1}^m U_i(i \ne k)\).

因此\(a_{k}=0\).由于\(k\)可以取遍每一个不为\(1\)的值,故\(a_2^1=\cdots=a_n^1=0\),即\(Tv_1=a_1^1 v_1\).

同理,\(\forall i=1,\cdots,m,Tv_i=a_i^i v_i\)均成立,可以推断任意非零向量均为\(T\)的特征向量.

根据\textit{5.A.26},\(T=\lambda I,\lambda \in F\).

\hspace*{\fill}

\textit{30.}
设\(T \in L(R^3)\)且\(-4,5,\sqrt{7}\)是\(T\)的特征值.
求证:存在\(x \in R^3\),使得\((T-9I)x=(-4,5,\sqrt{7})\).

\textit{Proof}
\(T \in L(R^3)\)最多拥有\(3\)个特征值,即\(9\)不是\(T\)的特征值.从而\(T-9I\)可逆,即满射,证毕.

\newpage

\textit{31.}
设\(V\)是有限维向量空间且\(v_1,\cdots,v_m \in V\).

求证:\(v_1,\cdots,v_m\)线性无关等价于\(v_1,\cdots,v_m\)是某\(T \in L(V)\)对应不同特征值的特征向量.

\textit{Proof}
必要性:根据定理5.10证毕.

充分性:由于\(v_1,\cdots,v_m\)线性无关,可以令\(v_1,\cdots,v_m,v_{m+1},\cdots,v_n\)为\(V\)的一组基.

定义\(Tv_i=iv_i,1=1,\cdots,n\)即可.

\hspace*{\fill}

\textit{32.}
设\(\lambda_1,\cdots,\lambda_n \in R\).
求证:\(e^{\lambda_1 x},\cdots,e^{\lambda_n x}\)在由实值函数组成的函数空间中线性无关.

\textit{Proof}
令\(V=\myspan (e^{\lambda_1 x},\cdots,e^{\lambda_n x})\),并定义\(T \in L(V)\)为\(Tf=f'\).

由于\((e^{\lambda_i x})'=\lambda_i e^{\lambda_i x}\),即\(Tf_i=\lambda_i f_i\),故\(e^{\lambda_i x}\)是\(T\)的特征向量.

根据定理5.10,\(e^{\lambda_1 x},\cdots,e^{\lambda_n x}\)线性无关.

\hspace*{\fill}

\textit{33.}
设\(T \in L(V)\).证明:\(T/(\Img T)=0\).

\textit{Proof}
\(\forall v \in V,(T/(\Img T))(v+(\Img T))=Tv+\Img T=\Img T\).

