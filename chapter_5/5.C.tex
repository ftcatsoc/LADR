\textit{1.}
设\(T \in L(V)\)可对角化.证明\(V=\Ker T \oplus \Img T\).

\textit{Proof}:
设\(\lambda_1,\cdots,\lambda_m\)是\(T\)的所有非零特征值.

由于\(T\)可对角化,故根据定理5.41,\(V=E(0,T)+\oplus_{i=1}^m E(\lambda_i,T)\).

考虑\(v_i \in E(\lambda_i,T)\),有\(Tv_i=\lambda_iv_i,v_i=\dfrac{1}{\lambda_i}Tv_i\).

因此\(T(E(\lambda_i,T)) \subseteq E(\lambda_i,T),E(\lambda_i,T) \subseteq T(E(\lambda_i,T))\).

即\(\Img T=\oplus_{i=1}^m E(\lambda_i,T)\)且\(\Ker T=E(0,T)\),证毕.

\hspace*{\fill}

\textit{3.}
设\(V\)是有限维向量空间且\(T \in L(V)\).证明以下三个命题等价:

(\textit{a})\(V=\Ker T \oplus \Img T\) \quad
(\textit{b})\(V=\Ker T+\Img T\) \quad
(\textit{c})\(\Ker T \cap \Img T=\{0\}\)

\textit{Proof}:
下证\(a \Rightarrow b \Rightarrow c \Rightarrow a\).
\(a \Rightarrow b\)是显然的.

假设\(b\)成立.根据定理2.43和3.22,
    \begin{align*}
        \mydim V &= \mydim \Ker T+\mydim \Img T \\
        \mydim V &= \mydim \Ker T+\mydim \Img T-\mydim (\Img T \cap \Ker T)
    \end{align*}
得到\(\mydim (\Img T \cap \Ker T)=0 \Rightarrow \Ker T \cap \Img T=\{0\}\).

假设\(c\)成立,令\(u_1,\cdots,u_m\)和\(w_1,\cdots,w_n\)分别为\(\Ker T\)和\(\Img T\)的一组基.

根据\textit{2.B.8},\(u_1,\cdots,u_m,w_1,\cdots,w_n\)是\(T\)的一组基.
从而\(V=\Ker T \oplus \Img T\),证毕.

\hspace*{\fill}

\textit{5.}
设\(V\)是有限维复向量空间且\(T \in L(V)\).设\(\lambda_1,\cdots,\lambda_m\)是\(T\)的不同特征值.

证明:若\(\forall i=1,\cdots,m,V=E(\lambda_i,I) \oplus \Img(T-\lambda_i I)\),则\(T\)可对角化.

\textit{Proof}:
使用数学归纳法,当\(\mydim V=1\)时,\(V=E(\lambda_1,T),\Img(T-\lambda_1 I)=\{0\}\),显然成立.

现假设对于任意的\(W\)满足\(\mydim W<\mydim V\),结论均成立.

由于\(\Img(T-\lambda_1 I)\)是\(T\)下的不变子空间且\(\mydim \Img(T-\lambda_1 I)<\mydim V\),故可以对其使用归纳.

令\(\Img(T-\lambda_1 I)=U\).\(T|_U\)的特征值为\(\lambda_2,\cdots,\lambda_m\),
故\(U=\oplus_{i=2}^m E(\lambda_i,T|_U)\).

以下将证明\(\forall i=2,\cdots,m,E(\lambda_i,T) \subseteq E(\lambda_i,T|_U)\),
设\(\forall i=2,\cdots,m,v_i^\alpha \in E(\lambda_i,T) \subseteq V\).

由于\(V=E(\lambda_1,T) \oplus U=E(\lambda_1,T) \oplus_{j=2}^m E(\lambda_i,T|_U)\),

故\(\exists v_j^\beta \in E(\lambda_i,T|_U),v_i^\alpha=v_1^\alpha+\sum_{j=2}^m v_j^\beta\).
因而\(v_i^\alpha-v_i^\beta=v_1^\alpha+\sum_{j=2}^m v_j^\beta(j \ne i)\).

根据定理5.10,若这些向量为\(T\)的特征向量,那么它们必须线性无关.

然而它们的系数均为\(1\),因此它们只能为零向量,这表明\(v_i^\alpha-v_i^\beta=v_1^\alpha=v_j^\beta=0\).

因此\(v_i^\alpha=v_i^\beta \in E(\lambda_i,T|_U)\),即\(E(\lambda_i,T) \subseteq E(\lambda_i,T|_U)\).

结合\(E(\lambda_i,T|_U) \subseteq E(\lambda_i,T)\),得到\(E(\lambda_i,T|_U)=E(\lambda_i,T)\).

于是\(U=\oplus_{i=2}^m E(\lambda_i,T),V=E(\lambda_1,T) \oplus U=\oplus_{i=1}^m E(\lambda_i,T)\),证毕.

\newpage

\textit{6.}
设\(V\)是有限维向量空间.

\(T \in L(V)\)有\(\mydim V\)个不同的特征向量,\(S \in L(V)\)有和\(T\)相同的特征向量.
求证:\(ST=TS\).

\textit{Proof}:
令\(\mydim V=n\),设\(v_1,\cdots,v_n\)是\(S,T\)的特征向量.

根据特征向量的独立性,\(v_1,\cdots,v_n\)是\(V\)的一组基.

设\(\forall i=1,\cdots,m,Tv_i=\lambda_i^\alpha v_i,Sv_i=\lambda_i^\beta v_i\).
    \begin{align*}
        (ST)v &=(ST)\sum_{i=1}^n a_iv_i=S\sum_{i=1}^n a_i \lambda_i^\alpha v_i
                =\sum_{i=1}^n a_i \lambda_i^\beta \lambda_i^\alpha v_i \\
        (TS)v &=(TS)\sum_{i=1}^n a_iv_i=T\sum_{i=1}^n a_i \lambda_i^\beta v_i
                =\sum_{i=1}^n a_i \lambda_i^\alpha \lambda_i^\beta v_i
    \end{align*}
显然\(ST=TS\).

\hspace*{\fill}

\textit{12.}
设\(V\)是有限维向量空间且\(\mydim V=n\).\(R,T \in L(V)\)都有特征值\(\lambda_1,\cdots,\lambda_n\).

求证:存在可逆变换\(S \in L(V)\),使得\(R=S^{-1}TS\).

\textit{Proof}:
根据定理5.44,显然\(R,T\)均可对角化.

设\(v_1^\alpha,\cdots,v_n^\alpha\)和\(v_1^\beta,\cdots,v_n^\beta\)
分别是\(R,T\)与\(\lambda_i\)对应的特征向量.

根据定理5.10和2.39,\(v_1^\alpha,\cdots,v_n^\alpha\)和\(v_1^\beta,\cdots,v_n^\beta\)分别是\(V\)的一组基.

定义线性变换\(S\)为\(Sv_i^\alpha=v_i^\beta,i=1,\cdots,n\),故有
    \begin{align*}
        (S^{-1}TS)v &=(S^{-1}TS)\sum_{i=1}^n a_iv_i^\alpha=(S^{-1}T)\sum_{i=1}^n a_iv_i^\beta \\
                    &=S^{-1}\sum_{i=1}^n a_i\lambda_i v_i^\beta=\sum_{i=1}^n a_i\lambda_i v_i^\alpha
                     =\sum_{i=1}^n a_iRv_i^\alpha=Rv
    \end{align*}
因而\(R=S^{-1}TS\),证毕.

\newpage

\textit{16.}
{\kaishu 斐波那契数列}\(F_1,F_2,\cdots\)定义如下:
    \begin{align*}
        F_1=1,F_2=1,F_n=F_{n-2}+F_{n-1},n \geq 3
    \end{align*}
并定义\(T \in L(R^2)\)为\(T(x,y)=(y,x+y)\).

(\textit{a})证明\(\forall n \in N^+,T^n(0,1)=(F_n,F_{n+1})\).

(\textit{b})给出\(T\)的特征值.

(\textit{c})给出一组以\(T\)的特征向量所构成的\(R^2\)的基.

(\textit{d})利用(\textit{c})部分的结论给出\(T^n(0,1)\),并证明
    \begin{align*}
        F_n=\dfrac{1}{\sqrt{5}}[(\dfrac{1+\sqrt{5}}{2})^n-(\dfrac{1-\sqrt{5}}{2})^n]
    \end{align*}
(\textit{e})利用(\textit{d})部分的结论证明{\kaishu 斐波那契数}\(F_n\)是最接近
\(\dfrac{1}{\sqrt{5}}(\dfrac{1+\sqrt{5}}{2})^n\)的整数.

(\textit{a.Proof}):
使用数学归纳法.\(n=1\)时,\(T(0,1)=(1,1)=(F_1,F_2)\).

下设\(n=k\)时结论成立.则\(n=k+1\)时,
    \begin{align*}
        T^{k+1}(0,1)=T(F_k,F_{k+1})=(F_{k+1},F_k+F_{k+1})=(F_{k+1},F_{k+2})
    \end{align*}
(\textit{b.Proof}):
令\(T(x,y)=\lambda(x,y)=(y,x+y)\),则\(y=\lambda x\)且\(x+y=\lambda y\).

得到\(\lambda^2-\lambda-1=0\),故\(\lambda_1=\dfrac{1+\sqrt{5}}{2},\lambda_2=\dfrac{1-\sqrt{5}}{2}\).

(\textit{c.Proof}):
$T(1,\dfrac{1+\sqrt{5}}{2})=\dfrac{1+\sqrt{5}}{2}(1,\dfrac{1+\sqrt{5}}{2}),
T(1,\dfrac{1-\sqrt{5}}{2})=\dfrac{1-\sqrt{5}}{2}(1,\dfrac{1+\sqrt{5}}{2})$.

(\textit{d.Proof}):
设\(v_1=(1,\dfrac{1+\sqrt{5}}{2}),v_2=(1,\dfrac{1-\sqrt{5}}{2})\).

对于\(\forall v \in R^2,\exists a_1,a_2 \in R,v=a_1v_1+a_2v_2\),

从而\(Tv=a_1\lambda_1 v_1+a_2\lambda_2 v_2,T^n v=a_1\lambda_1^n v_1+a_2\lambda_2^n v_2\).

令\(v=(0,1),v=a_1v_1+a_2v_2=(a_1+a_2,\dfrac{1+\sqrt{5}}{2}a_1+\dfrac{1-\sqrt{5}}{2}a_2)\).

得到\(a_1=\dfrac{1}{\sqrt{5}},a_2=-\dfrac{1}{\sqrt{5}}\).
根据\textit{5.C.16.a},\(T^n(0,1)=(F_n,F_{n+1})\).故
    \begin{align*}
        F_n=\dfrac{1}{\sqrt{5}}[(\dfrac{1+\sqrt{5}}{2})^n-(\dfrac{1-\sqrt{5}}{2})^n]
    \end{align*}
(\textit{e.Proof}):
即证\(\abs{F_n-\dfrac{1}{\sqrt{5}}(\dfrac{1+\sqrt{5}}{2})^n}=\abs{\dfrac{1}{\sqrt{5}}(\dfrac{1-\sqrt{5}}{2})^n}\).

由于\(\abs{\dfrac{1-\sqrt{5}}{2}}<1\),故\(\abs{(\dfrac{1-\sqrt{5}}{2})^n}<1\).
得到\(\abs{\dfrac{1}{\sqrt{5}}(\dfrac{1-\sqrt{5}}{2})}^n<\dfrac{1}{\sqrt{5}}<\dfrac{1}{2}\).

因此两数的差始终小于\(\dfrac{1}{2}\),证毕.

