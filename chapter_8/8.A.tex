\textit{Theorem 5.10/8.13}{\kaishu 特征向量/广义特征向量的独立性}
设\(V\)是有限维向量空间,\(T \in L(V)\).

\(\lambda_1,\cdots,\lambda_m\)是\(T\)的特征值,\(v_1,\cdots,v_m\)是分别与之对应的(广义)特征向量.

求证:\(v_1,\cdots,v_m\)线性无关.

\textit{5.10 Proof}
令\(\sum_{i=1}^m a_iv_i=0\),下证\(a_1=\cdots=a_m=0\).

设\(k \in \{1,\cdots,m\}\).对等式两边施加算子\(\prod_{i=1}^m (T-\lambda_i I)(i \ne k)\).
    \begin{align*}
        0=a_{k}\prod_{i=1}^m (T-\lambda_i I) v_{k}=
        a_{k}\prod_{i=1}^m (\lambda_{k}-\lambda_i) v_{k}
    \end{align*}
由于这些特征值各不相同,故\(\forall i=1,\cdots,m,\lambda_i-\lambda_{k} \ne 0\).
从而只能有\(a_{k}=0\).

让\(k\)依次等于\(1,\cdots,m\),则\(a_1=\cdots=a_m=0\),证毕.

\textit{8.13 Proof}
令\(\sum_{i=1}^m a_iv_i=0\),下证\(a_1=\cdots=a_m=0\).

设\(k \in \{1,\cdots,m\}\).
令\(k\)为最大的使\((T-\lambda_{k})^k v_{k} \ne 0\)的自然数.

令\(v_0=(T-\lambda_{k})^k v_{k}\),则\((T-\lambda_{k})^{k+1} v_{k}=0\).
从而\(Tv_0=\lambda_{k}v_0\),进一步有
    \begin{align*}
        (T-\lambda I)v_0=(\lambda_{k}-\lambda_i)v_0 \Rightarrow 
        (T-\lambda I)^{\mydim V} v_0=(\lambda_{k}-\lambda_i)^{\mydim V}v_0
    \end{align*}
对等式两边施加算子\(\prod_{i=1}^m (T-\lambda_i I)^{\mydim V}(T-\lambda_{k})^k(i \ne k)\).
    \begin{align*}
        0=a_{k}\prod_{i=1}^m (T-\lambda_i I)^{\mydim V}(T-\lambda_{k})^k v_{k}
        =a_{k}\prod_{i=1}^m (\lambda_{k}-\lambda_i)^{\mydim V} v_0
    \end{align*}
由于这些特征值各不相同,故\(\forall i=1,\cdots,m,(\lambda_i-\lambda_{k})^{\mydim V} \ne 0\).
从而只能有\(a_{k}=0\).

两者的证明思路大体相似,但定理8.13需要多乘以一个算子来构造一个“狭义”特征向量,
以使得定理5.10所施加的算子可操作.

\hspace*{\fill}

\textit{Theorem 8.19}{\kaishu 幂零算子的矩阵}
设\(N \in L(V)\)是一个幂零算子,则存在\(V\)的一组基,

使得\(M(N)\)是对角线元素均为\(0\)的上三角矩阵.

\textit{Proof}
设\(\mydim V=n\),并令\(B_1=\{v_1^1,\cdots,v_{n_1}^1\}\)是\(\Ker N\)的一组基.

由于\(\forall i=1,\cdots,n,\Ker N^i \subseteq \Ker N^{i+1}\),故依次扩充该基为\(\Ker N^2,\cdots,\Ker N^n\)的一组基.

从\(\Ker N^{i-1}\)扩充至\(\Ker N^i\)时,所添加的向量组为\(B_i=\{v_{n_{i-1}+1}^i,\cdots,v_{n_i}^i\},i=1,\cdots,n\).

因\(\Ker N^i=\myspan(B_1,\cdots,B_i)\)且\(\Ker N^n=V\),故\(B_1,\cdots,B_n\)中向量顺序排列即得\(V\)的基.

由于\(N(\Ker N^{i+1}) \subseteq \Ker N^i\),因此\(N \myspan(B_i) \subseteq \myspan(B_1,\cdots,B_{i-1})\).

因此\(M(N,(B_1,\cdots,B_n))\)中,\(B_i\)列中的\(B_i,\cdots,B_n\)行元素均为\(0\).

因此在基\(B_1,\cdots,B_n\)下,\(\forall v \in \myspan(B_i),Nv\)都不会由对角线及其之下的分量构成.

以矩阵语言表述,也就是\(M(N)\)是对角线元素均为\(0\)的上三角矩阵.

\newpage

\textit{2.}
定义\(T \in L(C^2)\)为\(T(w,z)=(-z,w)\).给出\(T\)的所有广义特征空间.

\textit{Proof}
先给出\(T\)的所有特征值.令\(T(w,z)=\lambda(w,z)=(-z,w)\),得到\(\lambda_1=i,\lambda_2=-i\).

因此\((T-iI)^2(w,z)=-2(w-iz,z+iw),(T+iI)^2(w,z)=(w+iz,z-iw)\).

最后\(G(i,T)=\Ker (T-iI)^2=(iz,z),G(-i,T)=\Ker (T+iI)^2=(-iz,z)\).

\hspace*{\fill}

\textit{3.}
设\(T \in L(V)\)是可逆变换.证明:\(\forall \lambda \ne 0,G(\lambda,T)=G(\lambda^{-1},T^{-1})\).

\textit{Proof}
设\(\mydim V=n\),则\(G(\lambda,T)=\Ker (T-\lambda I)^n,G(\lambda^{-1},T^{-1})=\Ker (T^{-1}-\lambda^{-1} I)^n\).

使用数学归纳法.\(n=1\)时,根据\textit{5.C.9},\(E(\lambda,T)=E(\lambda^{-1},T^{-1})\),命题成立.

接着假设\(\Ker (T-\lambda I)^{n-1}=\Ker (T^{-1}-\lambda^{-1} I)^{n-1}\),考虑\(\forall v \in \Ker (T-\lambda I)^n\).

即\((T-\lambda I)^n v=(T-\lambda I)^{n-1}(T-\lambda I)v=0\).

利用归纳假设,有\((T-\lambda I)v \in \Ker (T-\lambda I)^{n-1}=\Ker (T^{-1}-\lambda^{-1} I)^{n-1}\).

根据定理5.20,有\((T^{-1}-\lambda^{-1} I)^{n-1}(T-\lambda I)v=(T-\lambda I)(T^{-1}-\lambda^{-1} I)^{n-1}v=0\).

利用\(n=1\)时的结论,有\((T^{-1}-\lambda^{-1} I)^{n-1}v \in \Ker (T-\lambda I)=\Ker (T^{-1}-\lambda^{-1} I)\).

得到\((T^{-1}-\lambda^{-1} I)(T^{-1}-\lambda^{-1} I)^{n-1}v=(T^{-1}-\lambda^{-1} I)^n v=0\),

因此\(v \in \Ker (T^{-1}-\lambda^{-1} I)^n \Rightarrow G(\lambda,T) \subseteq G(\lambda^{-1},T^{-1})\).

\(G(\lambda^{-1},T^{-1}) \subseteq G(\lambda,T)\)的过程完全一致.综上,\(G(\lambda,T)=G(\lambda^{-1},T^{-1})\).

\hspace*{\fill}

\textit{5.}
设\(T \in L(V),v \in V,m \in N^*\),且满足\(T^{m-1}v \ne 0,T^m v=0\).

求证:\(v,Tv,\cdots,T^{m-1}v\)线性无关.

\textit{Proof}
令\(\sum_{i=0}^{m-1} a_iT^i v=0\).对等式两边施加算子\(T^{m-1}\),

得到\(\sum_{i=0}^{m-1} a_iT^{i+m-1} v=a_0T^{m-1} v=0\),因此\(a_0=0\).

接着依次对该等式两边施加算子\(T^{m-2},\cdots,T\),得到\(a_0=a_1=\cdots=a_{m-2}=0\).

而\(T^{m-1}v \ne 0\),因此必有\(a_{m-1}=0\),从而\(a_0=a_1=\cdots=a_{m-1}=0\),证毕.

\hspace*{\fill}

\textit{9.}
设\(S,T \in L(V)\)且\(ST\)是幂零算子,证明\(TS\)也是幂零算子.

\textit{Proof}
\(\Ker (TS)^{\mydim V}=\Ker (TS)^{\mydim V+1}=\Ker T(ST)^{\mydim V}S=V\).

其中\((ST)^{\mydim V}=0\),因此第三个等号成立.

\hspace*{\fill}

\textit{11.}
证明或给出反例:若\(T \in L(V)\)且\(\mydim V=n\),则\(T^n\)可对角化.

\textit{Proof}
反例:设\(V=C^2,T(z_1,z_2)=(z_1,0)\),则\(T^2(z_1,z_2)=(z_1,0)\).

因此\(T^2\)只有一个特征值,无法对角化.

\newpage

\textit{12.}
设\(N \in L(V)\)满足存在\(V\)的一组基,使得\(M(N)\)是对角元素均为\(0\)的上三角矩阵.

求证:\(N\)是幂零算子.

\textit{Proof}
设\(v_1,\cdots,v_m\)是满足条件的一组基.

由于该矩阵是三角矩阵,则根据定理5.26有\(Nv_i \in \myspan (v_1,\cdots,v_i)\).

进一步,由于对角线元素均为\(0\),因此\(Nv_i \in \myspan (v_1,\cdots,v_{i-1})\).

因此\(\forall i=1,\cdots,m,N^m v_i=0\),即\(N\)是幂零算子,证毕.

\hspace*{\fill}

\textit{15.}
设\(N \in L(V)\)满足\(\Ker N^{\mydim V-1} \ne \Ker N^{\mydim V}\).

求证:\(N\)是幂零算子且\(\forall i=1,\cdots,\mydim V,\mydim \Ker N^i=i\).

\textit{Proof}
由于\(\Ker N^{\mydim V-1} \ne \Ker N^{\mydim V}\),
故\(\mydim \Ker N^i\)随\(i\)在\([1,\mydim V]\)上严格单调增长.

因此\(\mydim \Ker N<\cdots<\mydim \Ker N^{\mydim V} \leq \mydim V\).

得到\(\mydim \Ker N^{\mydim V}=\mydim V\),即\(N\)是幂零算子,且\(\mydim \Ker N^i=i\).

\hspace*{\fill}

\textit{16.}
设\(T \in L(V)\).求证:\(V=\Img T^0 \supset \Img T^1 \supset \cdots\)

\textit{Proof}
\(\forall v \in V,T^{i+1}v \in \Img T^{i+1},T^{i+1}v=T^i(Tv) \in \Img T^i\).

因此\(\forall i \in N^*,\Img T^{i+1} \subseteq \Img T^i\).特别地,\(T^0=I,\Img I=V\).

\hspace*{\fill}

\textit{17.}
设\(T \in L(V),m \in N^*\)满足\(\Img T^m=\Img T^{m+1}\).求证:\(\forall i \in N^*,\Img T^{m+i}=\Img T^m\).

\textit{Proof}
根据\textit{8.A.16},\(\forall i \in N^*,\Img T^{m+i} \subseteq \Img T^m\).
下证\(\forall i \in N^*,\mydim \Img T^{m+i}=\mydim \Img T^m\).

根据定理3.22,\(\forall i \in N^*,\mydim \Img T^i=\mydim V-\mydim \Ker T^i\),

故\(\mydim \Img T^m=\mydim \Img T^{m+1}\)表明\(\mydim \Ker T^m=\mydim \Ker T^{m+1}\),即\(\Ker T^m =\Ker T^{m+1}\).

得\(\forall i \in N^*,\mydim \Img T^{m+i}=\mydim V-\mydim \Ker T^{m+i}=\mydim V-\mydim \Ker T^m=\mydim \Img T^m\).

从而根据\textit{2.C.1},\(\forall i \in N^*,\Img T^{i+1}=\Img T^i\).

\hspace*{\fill}

\textit{18.}
设\(T \in L(V),\mydim V=n\).求证:\(\Img T^n=\Img T^{n+1}=\cdots\)

\textit{Proof}
根据定理3.22和定理8.4,\(\forall i>n \in N^*,\mydim \Img T^{i+1}=\mydim \Img T^i\).

根据\textit{2.C.1}和\textit{8.A.16},\(\forall i \in N^*,\Img T^{i+1}=\Img T^i\).

