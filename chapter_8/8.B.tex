\textit{Theorem 8.21}{\kaishu 广义特征空间分解}

设\(V\)是有限维复向量空间且\(T \in L(V)\),\(\lambda_1,\cdots,\lambda_m\)是\(T\)所有的不同特征值.求证:

(\textit{a})\(\forall i=1,\cdots,m,G(\lambda_i,T)\)是\(T\)下的不变子空间.

(\textit{b})\(\forall i=1,\cdots,m,(T-\lambda_i I)|_{G(\lambda_i,T)}\)是幂零算子.

(\textit{c})\(V\)有一组由\(T\)的广义特征向量组成的基.

(\textit{d})\(V=\oplus_{i=1}^m G(\lambda_i,T)\).

\textit{lemma/Theorem 8.20}
\(\Ker p(T)\)和\(\Img p(T)\)是\(T\)下的不变子空间.

\textit{Proof}
对\(\Ker p(T)\)和\(\Img p(T)\)逐一验证即可.

\(\forall v \in \Ker p(T),p(T)(Tv)=T(p(T)v)=0 \Rightarrow Tv \in \Ker p(T)\).

\(\forall v \in \Img p(T),\exists u \in V,p(T)u=v \Rightarrow Tv=T(p(T)u)=p(T)(Tu),Tv \in \Img p(T)\).

\textit{a.Proof}
考虑\(p(z)=(z-\lambda_i)^n\),从而\(\Ker p(T)=G(\lambda_i,T)\).

根据引理8.20,\(G(\lambda_i,T)\)是\(T\)下的不变子空间.

\textit{b.Proof}
\(\Ker (T-\lambda_i I)^n=G(\lambda_i,T)\),显然\((T-\lambda_i I)|_{G(\lambda_i,T)}\)是幂零算子.

\textit{c.Proof}
使用第二数学归纳法,设\(\mydim V=n\).当\(n=1\)时,\(V=G(\lambda,T)\),命题成立.

设对于任意有限维复向量空间\(U\)和\(R \in L(U)\)满足\(\mydim U<\mydim V\),结论均成立.

由\(V=\Ker (T-\lambda_1 I)^n \oplus \Img (T-\lambda_1 I)^n\),令\(U=\Img (T-\lambda_1 I)^n\).

由\(U\)是\(T\)下的不变子空间且满足\(\mydim U<\mydim V\),因而\(U\)有一组\(T|_U\)的广义特征基.

\(G(\lambda_1,T)\)作为\(V\)的广义特征子空间显然有\(T|_{G(\lambda_i,T)}\)的广义特征基.

将\(G(\lambda_1,T)\)和\(U\)的广义特征基合并,就得到了\(V\)由\(T\)的广义特征向量组成的基.

\textit{d.Proof}
\textit{8.21.c}中构造的每个广义特征向量均属于一个特定的广义特征子空间.

将这些向量按照所属空间重新标号为\(v_1^i,\cdots,v_{n_i}^i,i=1,\cdots,m\),上标\(i\)表示第\(i\)个特征值.

令\(U_i=\myspan(v_1^i,\cdots,v_{n_i}^i)\),则\(V=\oplus_{i=1}^m U_i,\mydim V=\sum_{i=1}^m \mydim U_i\).

同时\(\forall i=1,\cdots,m,U_i \subseteq G(\lambda_i,T)\).
于是\(\forall i=1,\cdots,m,\mydim U_i \leq \mydim G(\lambda_i,T)\).

由于广义特征子空间之和为直和,得到
    \begin{align*}
        \mydim \oplus_{i=1}^m G(\lambda_i,T)=\sum_{i=1}^m \mydim G(\lambda_i,T) \geq \sum_{i=1}^m \mydim U_i=\mydim V
    \end{align*}
于是只能取等,有\(\forall i=1,\cdots,m,\mydim U_i=\mydim G(\lambda_i,T)\),即\(U_i=G(\lambda_i,T)\).

从而\(V=\oplus_{i=1}^m U_i=\oplus_{i=1}^m G(\lambda_i,T)\),证毕.

\newpage

\textit{Theorem 8.33}{\kaishu 可逆算子的平方根}
设\(V\)是有限维复向量空间,\(T \in L(V)\)是可逆算子.

证明:\(T\)存在一个平方根\(R\),使得\(R^2=T\).

\textit{lemma/Theorem 8.31}
若\(N \in L(V)\)是幂零算子,则\(I+N\)有一个平方根\(R\).

\textit{Proof}
根据\(\sqrt{1+x}=\sum_{i=1}^\infty a_ix^i,a_0=1\),
可以猜想\(I+N\)的平方根具有类似形式
    \begin{align*}
        R=\sum_{i=1}^{m-1} a_iN^i,a_0=1,N^m=0
    \end{align*}
因此尝试对左右两边平方,得到
    \begin{align*}
        I+N = (\sum_{i=1}^{m-1} a_iN^i)^2 
            = I+2a_1N+\sum_{i=2}^{m-1} (2a_i+f(a_1,\cdots,a_{i-1}))N^i
    \end{align*}
显然\(a_1=\dfrac{1}{2}\).对于第\(i\)项,其中的\(f(a_1,\cdots,a_{i-1})\)是已知的.

因此只需解出\(a_i\),使得\(2a_i+f(a_1,\cdots,a_{i-1})=0\)即可.

\textit{Theorem.Proof}
设\(\lambda_1,\cdots,\lambda_m\)是\(T\)的所有不同特征值.

根据\textit{Theorem 8.21.b},\(N_i=(T-\lambda_i I)|_{G(\lambda_i,T)}\)是幂零算子.

因此可以将\(T|_{G(\lambda_i,T)}\)分解为
    \begin{align*}
        \forall i=1,\cdots,m,T|_{G(\lambda_i,T)}=\lambda_i(I+\dfrac{N_i}{\lambda_i})
    \end{align*}
由于\(T\)是可逆变换,因而\(\lambda_1,\cdots,\lambda_m \ne 0\),该等式始终有意义.

因此根据引理,\(T|_{G(\lambda_i,T)}\)有平方根\(R_i\).

由于\(V=\oplus_{i=1}^m G(\lambda_i,T)\),
因此\(\forall v \in V,\exists u_i \in G(\lambda_i,T),v=\sum_{i=1}^m u_i\).定义\(R\)为
    \begin{align*}
        Rv=\sum_{i=1}^m R_iu_i i=1,\cdots,m \Rightarrow
        R^2v=\sum_{i=1}^m R_i^2 u_i=\sum_{i=1}^m T|_{G(\lambda_i,T)}u_i=Tv
    \end{align*}
因此\(R\)是\(T\)的一个平方根.

\newpage

\textit{3.}
设\(T,S \in L(V)\),且\(S\)是可逆算子.证明:\(T\)和\(S^{-1}TS\)的相同特征值有相同的重数.

\textit{Proof}
根据\textit{5.A.15},\(T\)和\(S^{-1}TS\)拥有相同的特征值,设\(\lambda\)是其中之一.

现在考虑\((S^{-1}TS-\lambda I)^{\mydim V}\).根据\textit{5.B.5},
    \begin{align*}
        (S^{-1}TS-\lambda I)^{\mydim V}&=(S^{-1}TS-\lambda S^{-1}S)^{\mydim V}=(S^{-1}(TS-\lambda S))^{\mydim V} \\
        &=(S^{-1}(T-\lambda I)S)^{\mydim V}=S^{-1}(T-\lambda I)^{\mydim V}S
    \end{align*}
对于\(v \in \Ker G(\lambda,T)\),考虑\(S^{-1}v\),有
\((S^{-1}(T-\lambda I)^{\mydim V}S)(S^{-1}v)=S((T-\lambda I)^{\mydim V}v)=0\).

因此\(S^{-1}(G(\lambda,T)) \subseteq G(\lambda,S^{-1}TS)\),
同理\(S(G(\lambda,S^{-1}TS)) \subseteq G(\lambda,T)\).

从而\(\mydim G(\lambda,T)=\mydim G(\lambda,S^{-1}TS)\).

\hspace*{\fill}

\textit{5.}
设\(V\)是复向量空间且\(T \in L(V)\).

求证:\(T\)有由特征向量组成的基等价于\(T\)的所有广义特征向量都是特征向量.

\textit{Proof}
必要性:根据定理8.23,\(T\)有一组由广义特征向量组成的基.

而\(T\)的所有广义特征向量都是特征向量,从而\(T\)有由特征向量组成的基.

充分性:设\(\lambda_1,\cdots,\lambda_m\)是\(T\)的不同特征值.

根据定理5.41,\(T\)有由特征向量组成的基等价于\(V=\oplus_{i=1}^m E(\lambda_i,T)\).
结合\(V=\oplus_{i=1}^m G(\lambda_i,T)\),

并由定理8.13指出的广义特征向量的无关性,得\(\forall i=1,\cdots,m,G(\lambda_i,T)=E(\lambda_i,T)\).

即\(T\)的所有广义特征向量都是特征向量,证毕.

\hspace*{\fill}

\textit{7.}
设\(V\)是复向量空间且\(T \in L(V)\),求证:\(\forall T \in L(V),\exists S \in L(V),S^3=T\).

\textit{Proof}
参考引理8.31的证明,猜想\(I+N\)的立方根也具有形式
    \begin{align*}
        R=\sum_{i=1}^{m-1} a_iN^i,a_0=1,N^m=0
    \end{align*}
令\(R^3=I+N\),得到
    \begin{align*}
        I+N=I+3a_1N+\sum_{i=2}^{m-1} (2a_i+f(a_1,\cdots,a_{i-1}))N^i
    \end{align*}
得到\(a_1=\dfrac{1}{3}\),依次解出剩下的\(a_2,\cdots,a_{m-1}\)即可,
下设\(\lambda_1,\cdots,\lambda_m\)是\(T\)的不同特征值.

考虑幂零算子\(N_i=(T-\lambda_i I)|_{G(\lambda,T)}\),
将\(T|_{G(\lambda_i,T)}\)分解为\(\lambda_i(I+\dfrac{N_i}{\lambda_i})\).

从而\(\forall i=1,\cdots,m\),\(N_i\)都有立方根\(R_i\).

由于\(V=\oplus_{i=1}^m G(\lambda_i,T)\),
因此\(\forall v \in V,\exists u_i \in G(\lambda_i,T),v=\sum_{i=1}^m u_i\).

定义\(R\)为\(Rv=\sum_{i=1}^m R_iu_i\),则\(R\)是\(T\)的一个立方根,证毕.

\newpage

\textit{10.}
设\(V\)是有限维复向量空间且\(T \in L(V)\).

求证:存在\(D,N \in L(V)\)满足\(T=D+N\)且\(D\)可对角化,\(N\)是幂零算子,\(DN=ND\).

\textit{Proof}
设\(\lambda_1,\cdots,\lambda_m\)是\(T\)的不同特征值,考虑\(T|_{G(\lambda_i,T)}\).

令\(D_i=\lambda_i I|_{G(\lambda_i,T)},N_i=(T-\lambda_i I)|_{G(\lambda_i,T)}\),
显然\(D_i\)可对角化,\(N_i\)是幂零算子.

根据定理8.21,\(V=\oplus_{i=1}^m G(\lambda_i,T)\).
因此\(\forall v \in V,\exists v_i \in G(\lambda_i,T),v=\sum_{i=1}^m v_i\).

分别定义\(D\)和\(N\)为\(Dv=\sum_{i=1}^m D_iv_i,Nv=\sum_{i=1}^m N_iv_i\).

\(M(D)\)只有对角线元素不为\(0\),即\(D\)可对角化;\(M(N)\)的对角线元素均为\(0\),即\(N\)是幂零算子.

下证\(DN=ND\).考虑\(\forall v \in V\),有
    \begin{align*}
        (ND)v=N\sum_{i=1}^m D_iv_i=\sum_{i=1}^m \lambda_iN_iv_i=(DN)v
    \end{align*}

\hspace*{\fill}

\textit{11.}
设\(V\)是有限维复向量空间且\(T \in L(V)\).

设\(v_1,\cdots,v_n\)是\(V\)的一组基,并满足\(M(T,(v_1,\cdots,v_n))\)是上三角矩阵.

求证:\(T\)的每个特征值\(\lambda\)在矩阵对角线上出现的次数即为\(\lambda\)作为\(T\)的特征值的重数.

\textit{Proof}
使用第二数学归纳法,设\(\mydim V=n\).当\(n=1\)时,\(V=G(\lambda,T)\),命题成立.

现假设对于任意有限维复向量空间\(U\)满足\(\mydim U<\mydim V=n\),结论均成立.

由\(V=\Ker (T-\lambda_1 I)^n \oplus \Img (T-\lambda_1 I)^n\),不妨令\(U=\Img (T-\lambda_1 I)^n\).

根据引理8.20,\(U\)是\(T\)下的不变子空间,且满足\(\mydim U<\mydim V\),因此可对\(U\)使用归纳假设.

故任意给定\(v_1,\cdots,w_p\)作为\(V\)的一组基使得\(M(T|_U,(v_1,\cdots,v_p))\)是上三角矩阵,

\(T|_U\)的所有特征值\(\lambda_2,\cdots,\lambda_m\)在矩阵对角线上出现的次数均为其重数.

考虑\(\Ker (T-\lambda_1 I)^n=G(\lambda_1,T)\).若有一组基使得\(M(T|_{G(\lambda_1,T)})\)是上三角矩阵,

则\(\lambda_1\)在矩阵对角线上出现的次数和\(\lambda_1\)的重数均为\(\mydim G(\lambda_1,T)\).

将\(T|_{G(\lambda_1,T)}\)上的基和\(T|_U\)上的基合并,即可得到\(M(T)\)的上三角矩阵.

此时对于任意\(i=1,\cdots,m\),\(\lambda_i\)在矩阵对角线上出现的次数即为\(\lambda\)作为\(T\)的特征值的重数.

