\textit{Theorem 8.46}{\kaishu 极小多项式的决定}
设\(V\)是有限维向量空间且\(T \in L(V)\).

设\(\lambda_1,\cdots,\lambda_m\)是\(T\)的不同特征值,
\(k_1,\cdots,k_m\)分别是\(\lambda_1,\cdots,\lambda_m\)对应的最大\textit{Jordan}块的维数.

求证:\(T\)的极小多项式是\(p_m(z)=\prod_{i=1}^m (z-\lambda_i)^{k_i}\).

\textit{Proof}:
先证\(\prod_{i=1}^m (T-\lambda_i I)^{k_i}=0\).根据定理8.21,\(V=\oplus_{i=1}^m G(\lambda_i,T)\),

因此分别考虑\(\forall i=1,\cdots,m,(T-\lambda_i I)^{k_i}|_{G(\lambda_i,T)}\),
并令\(N_i=(T-\lambda_i I)|_{G(\lambda_i,T)}\).

根据定理8.55,\(G(\lambda_i,T)\)存在一组\textit{Jordan}基\(N_i^{m_1}v_1,\cdots,v_1,\cdots,N_i^{m_n}v_n,\cdots,v_n\).

因此\(\max \{m_1,\cdots,m_n\}=k_i\),并令\(U_j=\myspan (N_i^{m_j}v_j,\cdots,v_j)\),
则\(G(\lambda_i,T)=\oplus_{j=1}^n U_j\).

根据定理8.55的证明,\(\forall j=1,\cdots,n,U_j\)都是\(N_i\)下的不变子空间,且有\(N_i^{m_j}|_{U_j}=0\).

因此\(\forall j=1,\cdots,n,N_i^{\max \{m_1,\cdots,m_n\}}|_{U_j}=0\),
即\((T-\lambda_i I)^{k_i}|_{G(\lambda_i,T)}=0\).

对于\(\prod_{i=1}^m (T-\lambda_i I)^{k_i}\),根据算子的可交换性,总是可以把因子\((T-\lambda_i I)^{k_i}\)移至最后,

从而\(\forall i=1,\cdots,m,\prod_{i=1}^m (T-\lambda_i I)^{k_i}|_{G(\lambda_i,T)}=0\),
即\(\prod_{i=1}^m (T-\lambda_i I)^{k_i}=0\).

下证其确为能使\(p(T)=0\)的幂次最低的首一多项式,考虑\(p'_m(z)=\prod_{i=1}^m (T-\lambda_i I)^{k'_i}\).

其中,\(\forall i=1,\cdots,m,k'_i \leq k_i\),且\(\exists r=1,\cdots,m,k'_r<k_r\).

由于\((T-\lambda_r I)\)的幂指数\(k'_r<k_r\),那么对于\(k_r\)所对应的\(U_j\),必然有\(N_r^{k'_r}|_{U_j} \ne 0\).

因此\(p_m(z)\)的幂次已然最低,即\(p_m(z)=\prod_{i=1}^m (z-\lambda_i)^{k_i}\)就是\(T\)的极小多项式.

\hspace*{\fill}

\textit{2.}
设\(V\)是有限维向量空间且\(T \in L(V)\)只有两个特征值\(5,6\).

求证:\((T-5I)^{n-1}(T-6I)^{n-1}=0\),其中\(n=\mydim V\).

\textit{Proof}:
\(T\)有两个特征值,故每个特征值的重数最多为\(n-1\).

因而\(T\)的特征多项式是\((T-5I)^{n-1}(T-6I)^{n-1}\)的因子,即\((T-5I)^{n-1}(T-6I)^{n-1}=0\).

\hspace*{\fill}

\textit{7.}
设\(V\)是有限维向量空间且\(P \in L(V)\)满足\(P^2=P\).

求证:\(P\)的特征多项式是\(z^m(z-1)^n\),其中\(m=\mydim \Ker P,n=\mydim \Img P\).

\textit{Proof}:
根据\textit{5.B.4},\(V=\Ker P \oplus \Img P\).

\(\forall u \in \Ker P,u \in E(0,T)\);
\(\forall Pv \in \Img P,P(Pv)=Pv \in \Img P\),即\(Pv \in E(1,T)\).

因此\(V=E(0,P) \oplus \Img P,\Img P \subseteq E(1,P) \subseteq G(1,P)\).

由空间维数的限制,只能有\(E(0,P)=G(0,P),\Img P=E(1,P)=G(1,P)\).

因此\(P\)的特征多项式为\(p_c(z)=z^{\mydim G(0,P)}(z-1)^{\mydim G(1,P)}=z^m(z-1)^n\).

\newpage

\textit{10.}
设\(V\)是有限维向量空间且\(T \in L(V)\)是可逆算子.

令\(p,q\)分别指代\(T,T^{-1}\)的特征多项式,证明:\(q(z)=\dfrac{1}{p(0)}z^{\mydim V}p(\dfrac{1}{z})\).

\textit{Proof}:
根据\textit{8.A.3}和\textit{5.A.21},

\(\lambda\)是\(T\)的特征值和\(\lambda^{-1}\)是\(T^{-1}\)的特征值等价,
且\(\mydim G(\lambda,T)=\mydim G(\lambda^{-1},T^{-1})\).

设\(\lambda_1,\cdots,\lambda_m\)是\(T\)的不同特征值,\(d_1,\cdots,d_m\)是其对应的重数,

则\(p(z)=\prod_{i=1}^m (z-\lambda_i)^{d_i},q(z)=\prod_{i=1}^m (z-\lambda_i^{-1})^{d_i}\).
    \begin{align*}
        q(z)=\prod_{i=1}^m (z-\lambda_i^{-1})^{d_i}
            =\prod_{i=1}^m \dfrac{z^{d_i}}{\lambda^{d_i}}(\lambda_i-\dfrac{1}{z})^{d_i}
            =\prod_{i=1}^m \dfrac{z^{d_i}}{-\lambda^{d_i}}(\dfrac{1}{z}-\lambda_i)^{d_i}
            =\dfrac{z^{\mydim V}}{p(0)}p(\dfrac{1}{z})
    \end{align*}

\hspace*{\fill}

\textit{12.}
设\(V\)是有限维向量空间且\(T \in L(V)\).

求证:\(T\)的极小多项式没有重根等价于\(T\)有由特征向量组成的基.

\textit{Proof}:
根据定理8.46,\(T\)的极小多项式\(p_m(z)=\prod_{i=1}^m (z-\lambda_i)^{k_i}\),

其中\(\lambda_1,\cdots,\lambda_m\)是\(T\)的不同特征值,
\(k_1,\cdots,k_m\)是\(\lambda_1,\cdots,\lambda_m\)对应的最大\textit{Jordan}块的维数.

由于\(p_m(z)\)没有重根,故\(k_1=\cdots=k_m=1\),因此\(G(\lambda_i,T)\)存在一组基\(v_1,\cdots,v_n\),

其中\((T-\lambda_i I)v_1=\cdots=(T-\lambda_i I)v_n=0\),
即\(v_1,\cdots,v_n \in \Ker (T-\lambda_i I)=E(\lambda_i,T)\),

因此\(T\)的所有广义特征向量都是特征向量.

根据\textit{8.B.5},这等价于\(T\)有由特征向量组成的基,证毕.

\hspace*{\fill}

\textit{18.}
设\(a_0,\cdots,a_{n-1} \in C\).给出以下矩阵的特征多项式和极小多项式.
    \begin{equation*}
        \begin{pmatrix}
            0 &   &        &   & -a_0     \\
            1 & 0 &        &   & -a_1     \\
              & 1 & \ddots &   & \vdots   \\
              &   & \ddots & 0 & -a_{n-2} \\
              &   &        & 1 & -a_{n-1} 
        \end{pmatrix}
    \end{equation*}
\textit{Proof}:
设\(e_1,\cdots,e_n\)是\(C^n\)的一组标准基.注意到\(\forall i=1,\cdots,n-1,Te_i=Te_{i+1}\).

因此\(T^n e_1=Te_n=-\sum_{i=0}^{n-1}a_ie_{i+1}=-\sum_{i=0}^{n-1}a_iT^ie_1\),
整理得\((\sum_{i=0}^{n-1}a_iT^i+T^n)e_1=0\).

从而矩阵的极小多项式为\(\sum_{i=0}^n a_iT^i,a_n=1\).

由于\(\mydim p_m(z)=\mydim V\),根据\textit{8.C.17},其特征多项式与极小多项式相同.

