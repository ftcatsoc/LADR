\textit{Theorem 8.55}\textit{Jordan}{\kaishu 基的存在性}
考虑有限维复向量空间\(V\)和幂零算子\(N \in L(V)\).

证明:\(V\)中存在向量组\(v_1,\cdots,v_n\)满足\(N^{m_1}v_1,\cdots,v_1,\cdots,N^{m_n}v_n,\cdots,v_n\)是\(V\)的一组基,

且\(N^{m_1+1}v_1=\cdots=N^{m_n+1}v_n=0\).

\textit{Proof}
使用数学归纳法.考虑\(\mydim V=1\)的情况.取\(\forall v \in V,m=0\),则\(v\)即为\(V\)的基.

现在假设对于任意的\(W\)满足\(\mydim U<\mydim V\),都存在一组如上形式的基.

任取\(v_1 \in V\),考虑一列向量\(v_1,\cdots,N^{m_1}v_1\),其中\(N^{m_1} \ne 0\)但\(N^{m_1+1}v_1=0\).

根据\textit{8.A.5},\(v_1,\cdots,N^{m_1}v_1\)线性无关.令\(U_1=\myspan (v_1,\cdots,N^{m_1}v_1)\),

则\(N(U_1)=\myspan (Nv_1,\cdots,N^{m_1}v_1) \subset U\),因此\(U_1\)是\(N\)下的不变子空间.

考虑商空间\(V/U_1,\mydim (V/U_1)=\mydim V-\mydim U_1<\mydim V\),因此可以对\(V/U_1\)使用归纳.

构造\(N\)在\(V/U_1\)上的诱导变换\(\ol{N} \in L(V/U_1)\)满足\(\forall \ol{v}=v+U_1 \in V/U_1,\ol{N}\ol{v}=Nv+U=\ol{Nv}\).

下面验证这个构造的合法性.考虑\(\ol{v_1}=\ol{v_2} \in V/U_1\),于是\(v_1-v_2 \in U_1\).

由于\(U_1\)是\(N\)下的不变子空间,因而\(N(v_1-v_2)=Nv_1-Nv_2 \in U_1\),即\(\ol{Nv_1}=\ol{Nv_2}\).

随后验证\(\ol{N}\)是线性变换,依旧考虑\(\ol{v_1},\ol{v_2} \in V/U_1\).
    \begin{align*}
        \ol{N}(c_1\ol{v_1}+c_2\ol{v_2})=&\ol{N(c_1v_1+c_2v_2)} =c_1\ol{Nv_1}+c_2\ol{Nv_2}
    \end{align*}
考虑\(\forall \ol{v} \in V/U_1\),则\(\ol{N}^m \ol{v}=\ol{N^m v}\).
由\(N\)是幂零算子得\(N^{\mydim V}=0,\ol{N^{\mydim V}}=\ol{0}\).

因此\(\ol{N} \in L(V/U_1)\)是合法的、线性的幂零算子,可以使用归纳.

则\(\ol{N^{m_2}u_2},\cdots,\ol{u_2},\cdots,\ol{N^{m_n}u_n},\cdots,\ol{u_n}\)是\(V/U_1\)的基,
满足\(\ol{N^{m_2+1}u_2}=\cdots=\ol{N^{m_n+1}u_n}=\ol{0}\).

因此\(\forall i=2,\cdots,n,N^{m_i+1}u_i=x_i^0 \in U\).
构造\(x_i^0,\cdots,x_i^{m_i+1}\)满足\(\forall j=1,\cdots,m_i+1,Nx_i^j=x_i^{j-1}\).

令\(\forall i=2,\cdots,n,v_i=u_i-x_i^{m_i+1}\),则\(N^{m_i+1}v_i=N^{m_i+1}u_i-N^{m_i+1}x_i^{m_i+1}=N^{m_i+1}u_i-x_i^0=0\).

则\(U_i=\myspan(N^{m_i}v_i,\cdots,v_i)\)是\(v_i\)生成的循环子空间,且在\(N\)下不变.

考虑到\(\forall j=0,\cdots,m_i,x_i^j \in U\),
故\(\ol{N^jv_i}=\ol{N^j(u_i-x_i^{m_i+1})}=\ol{N^ju_i-x_i^{m_i-j+1}}=\ol{N^ju_i}\).

故选择\(\ol{N^{m_2}u_2},\cdots,\ol{u_2},\cdots,\ol{N^{m_n}u_n},\cdots,\ol{u_n}\)中的代表元
\(N^{m_1}v_2,\cdots,v_2,\cdots,N^{m_n}v_n,\cdots,v_n\).

根据\textit{3.E.13},\(N^{m_1}v_1,\cdots,v_1,\cdots,N^{m_n}v_n,\cdots,v_n\)即为\(V\)的一组基.

下面讨论\(\forall i=2,\cdots,n,x_i^{m_i+1} \in U\)的存在性.令\(x_i^0=\sum_{k=0}^{m_1}c_kN^kv_1\),则
    \begin{align*}
        N^{m_i+1}x_i^{m_i+1}=x_i^0=\sum_{k=0}^{m_1}c_kN^k v_1 \Rightarrow x_i^{m_i+1}=\sum_{k=m_i+1}^{m_1}c_kN^k v_1
    \end{align*}
若\(m_i+1<m_1\),上式自然成立,直接采用该调整公式即可;

若\(m_i+1 \geq m_1\),则\(x_i^{m_i+1}=0\),即\(N^{m_i+1}u_i=0\),此时\(v_i=u_i\),不需要调整.

因此\(\forall i=2,\cdots,n,v_i\)总存在,即\(x_i^{m_i+1}\)的存在性得到保证,归纳完成.

