\textit{Theorem 7.10}{\kaishu 伴随算子的矩阵}
\(e_1,\cdots,e_m,f_1,\cdots,f_n\)分别是\(V,W\)的规范正交基.

设\(T \in L(V,W)\),则在这两组基下的\(M(T)\)是\(M(T^*)\)的共轭转置,即\(\ol{M(T)^T}=M(T^*)\).

\textit{Proof}
考虑\(M(T)_{i,j}\).由于\(Te_j=\sum_{i=1}^n \ip{Te_j}{f_i}f_i\),从而\(M(T)_{i,j}=\ip{Te_j}{f_i}\).

再考虑\(M(T^*)_{j,i}\).由于\(T^*f_i=\sum_{j=1}^n \ip{T^*f_i}{e_j}e_j\),从而\(M(T^*)_{j,i}=\ip{T^*f_i}{e_j}\).
    \begin{align*}
        M(T^*)_{j,i}=\ip{T^*f_i}{e_j}=\ip{f_i}{Te_j}=\ol{\ip{Te_j}{f_i}}=\ol{M(T)_{i,j}}
    \end{align*}
于是\(\ol{M(T)^T}=M(T^*)\),证毕.

\hspace*{\fill}

\textit{2.}
设\(V\)是有限维内积空间且\(T \in L(V)\).
证明:\(\lambda\)是\(T\)的特征值和\(\ol{\lambda}\)是\(T^*\)的特征值等价.

\textit{Proof}
利用\(\mydim \Ker(T^*)=\mydim \Ker(T)\),有
    \begin{align*}
        0<\mydim \Ker(T-\lambda I)=\mydim \Ker(T-\lambda I)^*=\mydim \Ker(T^*-\ol{\lambda}I)
    \end{align*}
因此\(E(\ol{\lambda},T^*) \ne \{0\}\),即\(\ol{\lambda}\)是\(T^*\)的一个特征值.

\hspace*{\fill}

\textit{3.}
设\(V\)是有限维内积空间且\(T \in L(V)\),\(U\)是\(V\)的一个子空间.

求证:\(U\)在\(T\)下不变和\(U^\bot\)和\(T^*\)下不变等价.

\textit{Proof}
由\(\forall u \in U,u' \in U^\bot\),则\(\ip{u}{u'}=0\),考虑\(\ip{Tu}{u'}\)和\(\ip{u}{T^*u'}\).

由\(\ip{Tu}{u'}=0 \Leftrightarrow \ip{u}{T^*u'}=0\),得到\(Tu \in U\)和\(T^*u' \in U^\bot\)等价.

\hspace*{\fill}

\textit{4.}
设\(T \in L(V,W)\).求证:
\textit{a.}\(T\)单射和\(T^*\)满射等价. \quad \textit{b.}\(T\)满射和\(T^*\)单射等价.

\textit{a.Proof}
\(\Ker T=\{0\} \Leftrightarrow (\Ker T)^\bot=V \Leftrightarrow \Img T^*=V\).

\textit{b.Proof}
\(\Img T=V \Leftrightarrow (\Img T)^\bot=\{0\} \Leftrightarrow \Img T^*=\{0\}\).

\hspace*{\fill}

\textit{11.}
设\(P \in L(V)\)满足\(P^2=P\).求证:\(T\)是自伴算子等价于存在\(V\)的子空间\(U\),使得\(P=P_U\).

\textit{Proof}
必要性:考虑\(U\)和\(U^\bot\).\(\forall v,w \in V,v=v_1+v_2,w=w_1+w_2,v_1,w_1 \in U,v_2,w_2 \in U^\bot\).
    \begin{align*}
        \ip{Pv}{w}=\ip{v_1}{w_1+w_2}=\ip{v_1}{w_1}+\ip{v_1}{w_2}=\ip{v_1}{w_1}+\ip{v_2}{w_1}=\ip{v}{Pw}
    \end{align*}
充分性:根据\textit{5.B.4},\(V= \Ker P \oplus \Img P\).
考虑\(\forall v=v_1+v_2,v_1 \in \Ker P,v_2 \in \Img P\).
    \begin{align*}
        \ip{Pv}{v}=\ip{v_2}{v_1+v_2}=\ip{v_2}{v_1}+\norm{v_2}^2
        =\ip{v_1}{v_2}+\norm{v_2}^2=\ip{v_1+v_2}{v_2}=\ip{v}{Pv}
    \end{align*}
因此\(\ip{v_1}{v_2}=0\),即\(\Ker P\)和\(\Img P\)互为正交补,取\(U=\Img P\)即可.

\newpage

\textit{14.}
设\(T \in L(V)\)是正规算子,且\(v,w \in V\)满足\(Tv=3v,Tw=4w,\norm{v}=\norm{w}=2\).

\textit{Proof}
根据定理7.22,\(\ip{v}{w}=0\),从而\(\norm{T(v+w)}^2=\norm{3v+4w}^2=\norm{3v}^2+\norm{4w}^2=100\).

根据范数的正定性,显然有\(\norm{T(v+w)}=10\).

\hspace*{\fill}

\textit{15.}
给定\(u,x \in V\),定义\(T \in L(V)\)为\(\forall v \in V,Tv=\ip{v}{u}x\).

(\textit{a})当\(F=R\)时,证明:\(T\)是自伴算子等价于\(\exists \lambda \ne 0 \in R,x=\lambda u\).

(\textit{b})证明:\(T\)是正规算子等价于\(\exists \lambda \ne 0,x=\lambda u\).

\textit{a.Proof}
    \begin{align*}
        \ip{v}{T^*w}=\ip{Tv}{w}=\ip{v}{u}\ip{x}{w}=\ip{v}{\ip{x}{w}u},T^*w=\ip{x}{w}u
    \end{align*}

必要性:\(\ip{Tv}{w}=\ip{\lambda \ip{v}{u}}{w}u=\lambda \ip{v}{u} \ip{u}{w}\)
\(=\ip{v}{\lambda \ip{w}{u}u}=\ip{v}{Tw}\).

充分性:\(\ip{x}{w}u=T^*w=Tw=\ip{w}{u}x\).取\(w=u\),得\(\lambda=\dfrac{\ip{x}{x}}{\ip{x}{u}}\).

\textit{b.Proof}
    \begin{align*}
        &(T^*T)v=T^*(\ip{v}{u}x)=\ip{v}{u}T^*x=\norm{x}^2\ip{v}{u}u \\
        &(TT^*)v=T(\ip{x}{v}u)=\ip{x}{v}Tu=\norm{u}^2\ip{x}{v}x
    \end{align*}
必要性:\((T^*T)v=\norm{\lambda u}^2 \ip{v}{u}u=\lambda \ol{\lambda} \norm{u}^2 \ip{v}{u}u\)
\(=\norm{u}^2 \ip{v}{\lambda u}\lambda u=(TT^*)v\).

充分性:\(\norm{x}^2\ip{v}{u}u=\norm{u}^2\ip{x}{v}x\).取\(v=x\),得\(\lambda=\dfrac{\ip{x}{u}}{\ip{u}{u}}\).

\hspace*{\fill}

\textit{16.}
设\(T \in L(V)\)是正规算子.求证:\(\Ker T^*=\Ker T,\Img T^*=\Img T\).

\textit{Proof}
\(\forall v \in V,Tv=0 \Leftrightarrow \norm{Tv}=0 \Leftrightarrow \norm{T^*v}=0 \Leftrightarrow T^*v=0\).

\(\Img T^*=(\Ker T)^\bot=(\Ker T^*)^\bot=\Img T\).

\(V=\Ker T \oplus (\Ker T)^\bot=\Ker T \oplus \Img T^*=\Ker T \oplus \Img T\).

\hspace*{\fill}

\textit{17.}
设\(T \in L(V)\)是正规算子.求证:\(\forall n \in N^*,\Ker T^n=\Ker T,\Img T^n=\Img T\).

\textit{Proof}
先证明\(\Ker T^2 \subseteq \Ker T\).考虑\(v \in \Ker T^2\),则\(\ip{T^2 v}{T^2 v}=0\).
    \begin{align*}
        &\ip{T^2 v}{T^2 v}=\ip{Tv}{T^*T^2v}=\ip{Tv}{T^2T^*v}=\ip{T^*Tv}{TT^*v}=0 \Rightarrow T^*Tv=0 \\
        &\ip{T^*Tv}{v}=\ip{Tv}{Tv}=0 \Rightarrow Tv=0 \Rightarrow v \in \Ker T
    \end{align*}
结合显然的\(\Ker T \subseteq \Ker T^2\)有\(\Ker T^2=\Ker T\).

根据定理8.3和\textit{8.A.17},有\(\forall n \in N^*,\Ker T^n=\Ker T,\Img T^n=\Img T\).

