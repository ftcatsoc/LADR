\textit{6.B.16.}
设\(V\)是有限维复内积空间,\(T \in L(V)\),\(\lambda_1,\cdots,\lambda_n\)是\(T\)的所有相异特征值.

\(T\)的谱半径\(\rho(T)<1\).证明:\(\forall \varepsilon>0,\exists m \in N^*,\norm{T^m}_2<\varepsilon\).

\textit{Proof}
\(V=\oplus_{i=1}^n G(\lambda_i,T)=\oplus_{i=1}^n\oplus_{j=1}^{n_i} J_j(\lambda_i,T)\),
令\(\mydim J_j(\lambda_i,T)=d_{i,j}\).

考虑\(M(T^m|_{J_j(\lambda_i,T)})\),展开之,得到
    \begin{align*}
        M(T^m|_{J_j(\lambda_i,T)})=
        \begin{pmatrix}
            \lambda_i^m & C_m^1 \lambda_i^{m-1} & \cdots      & C_m^{m+1-d_{i,j}} \lambda_i^{m+1-d_{i,j}} \\
                        & \lambda_i^m           & \ddots      & \vdots                                    \\
                        &                       & \lambda_i^m & C_m^1 \lambda_i^{m-1}                     \\
                 0      &                       &             & \lambda_i^m
        \end{pmatrix}
    \end{align*}
因此\(M(T^m|_{J_j(\lambda_i,T)})\)中的元素具有统一形式\(C_m^k \lambda_i^{m-k},k=0,\cdots,d_{i,j}-1\),考虑其绝对值.

显然,由于\(\abs{\lambda_i}<1\),故其最大值将在前半段取得.令\(a_k=C_m^k \abs{\lambda_i}^{m-k},k=0,\cdots,d_{i,j}-1\).令
    \begin{align*}
        \dfrac{a_{k+1}}{a_k}=\dfrac{C_m^{k+1}\abs{\lambda_i}^{m-(k+1)}}{C_m^k \abs{\lambda_i}^{m-k}}=
        \dfrac{m-k}{k+1} \abs{\lambda_i}^{-1}=1,k=\dfrac{m-\abs{\lambda_i}}{1+\abs{\lambda_i}}
    \end{align*}
因此\(a_{\max}\)在\([\dfrac{m-\abs{\lambda_i}}{1+\abs{\lambda_i}}]\)或\(\{\dfrac{m-\abs{\lambda_i}}{1+\abs{\lambda_i}}\}\)处取得.
随着\(m\)增长,将有\(\dfrac{m-\abs{\lambda_i}}{1+\abs{\lambda_i}}>d_{i,j}-1\).

因此当\(m\)足够大时,数列在\([1,d_{i,j}-1]\)上单调增,其最大值将在最后一项取得,为
    \begin{align*}
        a_{d_{i,j}-1}=C_m^{d_{i,j}-1}\abs{\lambda}^{m-(d_{i,j}-1)} \leq 
        \dfrac{m^{d_{i,j}-1}}{(d_{i,j}-1)!}\abs{\lambda}^{m-(d_{i,j}-1)}
    \end{align*}
选择最大行和范数\(\norm{\cdot}_{\infty}\)作为算子范数,因而
    \begin{align*}
        \norm{T^m|_{J_j(\lambda_i,T)}}_{\infty} \leq d_{i,j}\dfrac{m^{d_{i,j}-1}}{(d_{i,j}-1)!}\abs{\lambda}^{m-(d_{i,j}-1)}
        =\dfrac{d_{i,j}}{(d_{i,j}-1)! \abs{\lambda}^{d_{i,j}-1}}m^{d_{i,j}-1}\abs{\lambda_i}^m
    \end{align*}
由于\(\abs{\lambda_i}<1\),故\(\lim_{m \rightarrow +\infty}\norm{T^m|_{J_j(\lambda_i,T)}}_{\infty}=\)
\(\lim_{m \rightarrow +\infty}m^{d_{i,j}-1}\abs{\lambda_i}^m=0\).

令\(\sum_{i=1}^n n_i=N\),则\(\forall \varepsilon>0,\exists m \in N^*,\)
\(\norm{T^m|_{J_j(\lambda_i,T)}}_{\infty}<\dfrac{\varepsilon}{N \sqrt{d_{\max}}}\),因此
    \begin{align*}
        \norm{T^m}_{\infty} \leq \sum_{i=1}^n \sum_{j=1}^{n_i}\norm{T^m|_{J_j(\lambda_i,T)}}<
        N \cdot \dfrac{\varepsilon}{N \sqrt{d_{\max}}}=\dfrac{\varepsilon}{\sqrt{d_{\max}}}
    \end{align*}
由于\(\norm{T^m|_{J_j(\lambda_i,T)}}_2 \leq \sqrt{d_{i,j}}\norm{T^m|_{J_j(\lambda_i,T)}}_{\infty}\),
故\(\norm{T^m}_2 \leq \sqrt{d_{\max}}\norm{T^m}_{\infty}<\varepsilon\).

范数的等价性证明将于下页补齐.

\newpage

\textit{6.B.16.Lemma}
设\(V\)是有限维复向量空间,满足\(\mydim V=n,T \in L(V)\).

最大行和范数\(\norm{T}_{\infty}=\max_{1 \leq i \leq n} \sum_{j=1}^n \abs{M(T)_{i,j}}\)和谱范数\(\norm{T}_2\)满足
\(\norm{T}_2 \leq \sqrt{n}\norm{T}_{\infty}\).

\textit{Proof}
设\(e_1,\cdots,e_n\)是\(V\)的一组基.
下证\(\forall v=\sum_{i=1}^n \ip{v}{e_i}e_i \in V,\norm{Tv}^2 \leq n \norm{Tv}_{\infty}^2\).
    \begin{align*}
        \norm{Tv}^2=\abs{\sum_{j=1}^n \ip{v}{e_i}Tv_i}^2 \leq \sum_{i=1}^n \ip{v}{e_i}^2 \norm{Tv_i}^2 
        \leq n \norm{T}_{\infty}^2 \sum_{i=1}^n (\ip{v}{e_i}e_i)^2=n \norm{T}_{\infty}^2 \norm{v}^2
    \end{align*}
第一个不等号来自于柯西-施瓦兹不等式,第二个不等号来自于最大行和范数的定义.
    \begin{align*}
        \norm{T}_2^2=\sup_{v \ne 0} \dfrac{\norm{Tv}^2}{\norm{v}^2} \leq n \norm{T}_{\infty}^2,
        \norm{T}_2 \leq \sqrt{n}\norm{T}_{\infty}
    \end{align*}

