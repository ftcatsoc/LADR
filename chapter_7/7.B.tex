\textit{Theorem 7.24}{\kaishu 复谱定理}
设\(V\)是有限维复内积空间且\(T \in L(V),\mydim V=n\).

求证:\(T\)是正规算子等价于\(V\)有一组\(T\)的规范正交特征向量组成的基.

\textit{Proof}
必要性:若\(V\)有一组\(T\)的规范正交特征基,则\(M(T),M(T^*)\)都是对角矩阵.

由于对角矩阵的可交换性,有\(M(TT^*)=M(T)M(T^*)=M(T^*)M(T)=M(T^*T)\).

因此\(T^*T=TT^*\),\(T\)是正规算子.

充分性:根据定理6.38,\(V\)有一组规范正交基\(e_1,\cdots,e_n\)使得\(M(T)\)为上三角矩阵.

由于\(M(T^*)\)是\(M(T)\)的共轭转置,故\(M(T^*)\)是下三角矩阵.

由于\(T\)是正规算子,故根据定理7.20,\(i=1,\cdots,n,\norm{Te_i}^2=\norm{T^*e_i}^2\).设\(M(T)_{i,j}=a_{i,j}\).
    \begin{align*}
        \norm{Te_1}^2=\abs{a_{1,1}}^2=\sum_{i=1}^n \abs{a_{1,i}}^2=\norm{T^*e_1}^2,a_{2,1}=\cdots=a_{n,1}=0
    \end{align*}
进一步地,由\(\norm{Te_2}^2=\norm{T^*e_2}^2\)可以得到\(a_{3,2}=\cdots=a_{n,2}=0\).

以此类推,\(M(T)\)的所有严格上三角部分元素和严格下三角部分元素均为\(0\),即\(T\)可对角化.

此时的\(e_1,\cdots,e_n\)就是\(T\)的特征向量,它们构成了\(V\)的规范正交基.

\hspace*{\fill}

\textit{Lemma 7.28}{\kaishu 自伴算子和不变子空间}
设\(T \in L(V)\)是自伴算子且\(U\)是\(T\)下的不变子空间,则有

(\textit{a})\(U^\bot\)是\(T\)下的不变子空间. \quad
(\textit{b})\(T|_U\)是自伴算子. \quad
(\textit{c})\(T_{U^\bot}\)是自伴算子.

\textit{a.Proof}
设\(\forall u \in U,u' \in U^\bot\),由于\(Tu \in U\),故\(\ip{Tu}{u'}=\ip{u}{Tu'}=0\),即\(Tu' \in U^\bot\).

\textit{b.Proof}
设\(\forall u_1,u_2 \in U\),则\(\ip{T|_U u_1}{u_2}=\ip{Tu_1}{u_2}=\ip{u_1}{Tu_2}=\ip{u_1}{T|_U u_2}\).

\textit{c.Proof}
设\(\forall u_1,u_2 \in U^\bot\),则\(\ip{T|_{U^\bot}u_1}{u_2}=\ip{Tu_1}{u_2}=\ip{u_1}{Tu_2}=\ip{u_1}{T|_{U^\bot}u_2}\).

\hspace*{\fill}

\textit{Theorem 7.29}{\kaishu 实谱定理}
设\(V\)是有限维实内积空间且\(T \in L(V),\mydim V=n\).

求证:\(T\)是自伴算子等价于\(V\)有一组\(T\)的规范正交特征向量组成的基.

\textit{Proof}
必要性:若\(V\)有一组\(T\)的规范正交特征基,则\(M(T),M(T^*)\)都是对角矩阵.

由对角矩阵转置不变,有\(M(T)=M(T)^T=M(T^*)\).因此\(T\)是自伴算子.

充分性:由于实内积空间上的自伴算子一定有特征值\footnote{这个结论将在复化章节中证明.},

故设\(e_1\)满足\(\norm{e_1}=1\)是\(T\)的特征向量,令\(U=\myspan(e_1)\),考虑\(U^\bot\).

根据引理7.28,\(T|_{U^\bot}\)是自伴算子,故\(T|_{U^\bot}\)也有特征向量\(e_2\)满足\(\norm{e_2}=1\).

以此类推,\(T\)将有特征向量\(e_1,\cdots,e_n\)满足\(\forall i=1,\cdots,n,\norm{e_i}=1\).

且这些特征向量两两正交,于是得到了\(V\)的一组规范正交特征基.

