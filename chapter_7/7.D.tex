\textit{Lemma 7.64*}{\kaishu 奇异值分解的引理}
设\(V\)是有限维复内积空间且\(T \in L(V)\).则有

(\textit{a})\(T^*T \in L(V),TT^* \in L(W)\)是正算子. \quad
(\textit{b})\(\Ker T^*T=\Ker T\). \quad 
(\textit{c})\(\Img T^*T=\Img T^*\).

\textit{a.Proof}
由于\((T^*T)^*=T^*(T^*)^*=T^*T\),即\(T^*T\)为自伴算子.

从而\(\forall v \in V,\ip{T^*Tv}{v}=\ip{Tv}{Tv}=\norm{Tv}^2 \geq 0\),即\(T^*T\)为正算子.\(TT^*\)同理.

\textit{b.Proof}
设\(v \in \Ker T^*T\),则\(\ip{T^*Tv}{v}=\ip{Tv}{Tv}=\norm{Tv}^2=0\),即\(\Ker T^*T \subseteq \Ker T\).

\(\Ker T \subseteq \Ker T^*T\)是显然的.结合之,即有\(\Ker T^*T=\Ker T\).

\textit{c.Proof}
\(\Img T^*T=(\Ker (T^*T)^*)^\bot=(\Ker T^*T)^\bot=(\Ker T)^\bot=\Img T^*\).

\hspace*{\fill}

\textit{Theorem 7.70*}{\kaishu 奇异值分解}
设\(V\)是有限维复内积空间且\(T \in L(V)\),其奇异值为\(s_1,\cdots,s_n\).

那么存在\(V\)的两组规范正交基\(e_1,\cdots,e_n\)和\(f_1,\cdots,f_n\),
使得\(\forall v \in V,Tv=\sum_{i=1}^n s_i\ip{v}{e_i}f_i\).

\textit{Proof}
\(T^*T\)作为自伴算子有一组规范正交特征基\(e_1,\cdots,e_n\)满足\(\forall i=1,\cdots,n,T^*Tv=s_i^2 v\).

定义\(\forall i=1,\cdots,n,f_i\)为\(\dfrac{1}{s_i}Te_i\),下证\(f_1,\cdots,f_n\)为\(V\)的规范正交基.
    \begin{align*}
        \ip{f_j}{f_k}=\dfrac{1}{s_js_k}\ip{Te_j}{Te_k}=\dfrac{1}{s_js_k}\ip{T^*Te_j}{e_k}
        =\dfrac{1}{s_js_k}\ip{s_j^2 e_j}{e_k}=\dfrac{s_j}{s_k}\ip{e_j}{e_k}=\delta_{j,k}
    \end{align*}
其中\(\delta_{i,k}\)是\textit{Kronecker}函数.因此,\(f_1,\cdots,f_n\)确实为\(V\)的规范正交基.
    \begin{align*}
        Tv=T\sum_{i=1}^n \ip{v}{e_i}e_i=\sum_{i=1}^n \ip{v}{e_i}Te_i
        =\sum_{i=1}^n \ip{v}{e_i}s_if_i=\sum_{i=1}^n s_i\ip{v}{e_i}f_i
    \end{align*}

\hspace*{\fill}

\textit{Theorem 7.93*}{\kaishu 极分解}
设\(T \in L(V)\),则存在等距映射\(S \in L(V)\)使得\(T=S\sqrt{T^*T}\).

\textit{Proof}
沿用\textit{Theorem 7.51}证明中的记号,设\(\forall v \in V,Tv=\sum_{i=1}^n s_i\ip{v}{e_i}f_i\).

定义\(S \in L(V)\)为\(Sv=\sum_{i=1}^n \ip{v}{e_i}f_i\),下面验证\(S\)是等距映射.
    \begin{align*}
        \norm{Sv}^2=\norm{\sum_{i=1}^n \ip{v}{e_i}f_i}^2=\sum_{i=1}^n \norm{\ip{v}{e_i}f_i}^2
        =\sum_{i=1}^n \norm{\ip{v}{e_i}}^2=\norm{v}^2
    \end{align*}
其中第二个等号来自定理6.25,第四个等号来自定理6.30.

由\(T^*v=\sum_{i=1}^n s_i\ip{v}{f_i}e_i\)得到\(T^*Tv=\sum_{i=1}^n s_i^2 \ip{v}{e_i}e_i\),
因而\(\sqrt{T^*T}v=\sum_{i=1}^n s_i\ip{v}{e_i}e_i\).
    \begin{align*}
        S\sqrt{T^*T}v=S\sum_{i=1}^n s_i\ip{v}{e_i}e_i=\sum_{i=1}^n s_i\ip{v}{e_i}Se_i
        =\sum_{i=1}^n s_i\ip{v}{e_i}f_i=Tv
    \end{align*}

