\textit{Lemma 7.64*}{\kaishu 奇异值分解的引理}
设\(V\)是有限维复内积空间且\(T \in L(V)\).则有

(\textit{a})\(T^*T \in L(V),TT^* \in L(W)\)是正算子. \quad
(\textit{b})\(\Ker T^*T=\Ker T\). \quad 
(\textit{c})\(\Img T^*T=\Img T^*\).

\textit{a.Proof}
由于\((T^*T)^*=T^*(T^*)^*=T^*T\),即\(T^*T\)为自伴算子.

从而\(\forall v \in V,\ip{T^*Tv}{v}=\ip{Tv}{Tv}=\norm{Tv}^2 \geq 0\),即\(T^*T\)为正算子.\(TT^*\)同理.

\textit{b.Proof}
设\(v \in \Ker T^*T\),则\(\ip{T^*Tv}{v}=\ip{Tv}{Tv}=\norm{Tv}^2=0\),即\(\Ker T^*T \subseteq \Ker T\).

\(\Ker T \subseteq \Ker T^*T\)是显然的.结合之,即有\(\Ker T^*T=\Ker T\).

\textit{c.Proof}
\(\Img T^*T=(\Ker (T^*T)^*)^\bot=(\Ker T^*T)^\bot=(\Ker T)^\bot=\Img T^*\).

\hspace*{\fill}

\textit{Theorem 7.70*}{\kaishu 奇异值分解}
设\(V\)是有限维复内积空间且\(T \in L(V)\),其奇异值为\(s_1,\cdots,s_n\).

那么存在\(V\)的两组规范正交基\(e_1,\cdots,e_n\)和\(f_1,\cdots,f_n\),
使得\(\forall v \in V,Tv=\sum_{i=1}^n s_i\ip{v}{e_i}f_i\).

\textit{Proof}
\(T^*T\)作为自伴算子有一组规范正交特征基\(e_1,\cdots,e_n\)满足\(\forall i=1,\cdots,n,T^*Tv=s_i^2 v\).

定义\(\forall i=1,\cdots,n,f_i\)为\(\dfrac{1}{s_i}Te_i\),下证\(f_1,\cdots,f_n\)为\(V\)的规范正交基.
    \begin{align*}
        \ip{f_j}{f_k}=\dfrac{1}{s_js_k}\ip{Te_j}{Te_k}=\dfrac{1}{s_js_k}\ip{T^*Te_j}{e_k}
        =\dfrac{1}{s_js_k}\ip{s_j^2 e_j}{e_k}=\dfrac{s_j}{s_k}\ip{e_j}{e_k}=\delta_{j,k}
    \end{align*}
其中\(\delta_{i,k}\)是\textit{Kronecker}函数.因此,\(f_1,\cdots,f_n\)确实为\(V\)的规范正交基.
    \begin{align*}
        Tv=T\sum_{i=1}^n \ip{v}{e_i}e_i=\sum_{i=1}^n \ip{v}{e_i}Te_i
        =\sum_{i=1}^n \ip{v}{e_i}s_if_i=\sum_{i=1}^n s_i\ip{v}{e_i}f_i
    \end{align*}
因此\(M(T,(e_1,\cdots,e_n),(f_1,\cdots,f_n))=\diag(s_1,\cdots,s_n)\).

\hspace*{\fill}

\textit{Theorem 7.93*}{\kaishu 极分解}
设\(T \in L(V)\),则存在等距映射\(S \in L(V)\)使得\(T=S\sqrt{T^*T}\).

\textit{Proof}
沿用\textit{Theorem 7.70*}证明中的记号,设\(\forall v \in V,Tv=\sum_{i=1}^n s_i\ip{v}{e_i}f_i\).

定义\(S \in L(V)\)为\(Sv=\sum_{i=1}^n \ip{v}{e_i}f_i\),下面验证\(S\)是等距映射.
    \begin{align*}
        \norm{Sv}^2=\norm{\sum_{i=1}^n \ip{v}{e_i}f_i}^2=\sum_{i=1}^n \norm{\ip{v}{e_i}f_i}^2
        =\sum_{i=1}^n \norm{\ip{v}{e_i}}^2=\norm{v}^2
    \end{align*}
其中第二个等号来自定理6.25,第四个等号来自定理6.30.

由\(T^*v=\sum_{i=1}^n s_i\ip{v}{f_i}e_i\)得到\(T^*Tv=\sum_{i=1}^n s_i^2 \ip{v}{e_i}e_i\),
因而\(\sqrt{T^*T}v=\sum_{i=1}^n s_i\ip{v}{e_i}e_i\).
    \begin{align*}
        S\sqrt{T^*T}v=S\sum_{i=1}^n s_i\ip{v}{e_i}e_i=\sum_{i=1}^n s_i\ip{v}{e_i}Se_i
        =\sum_{i=1}^n s_i\ip{v}{e_i}f_i=Tv
    \end{align*}

\newpage

\textit{1.}
给定\(u \ne 0,x \in V\),定义\(T \in L(V)\)为\(\forall v \in V,Tv=\ip{v}{u}x\).
证明:\(\sqrt{T^*T}v=\dfrac{\norm{x}}{\norm{u}}\ip{v}{u}u\).

\textit{Proof}
根据\textit{7.A.15},\((T^*T)v=T^*(\ip{v}{u}x)=\ip{v}{u}T^*x=\norm{x}^2\ip{v}{u}u\).

令\(Rv=\dfrac{\norm{x}}{\norm{u}}\ip{v}{u}u\),验证\(R\)是\(T^*T\)的平方根即可.
    \begin{align*}
        R^2v=\dfrac{\norm{x}}{\norm{u}}\ip{\dfrac{\norm{x}}{\norm{u}}\ip{v}{u}u}{u}u
        =\dfrac{\norm{x}}{\norm{u}}\dfrac{\norm{x}}{\norm{u}}\ip{v}{u}\norm{u}^2u
        =\dfrac{\norm{x}}{\norm{u}}\ip{v}{u}u=Tv
    \end{align*}

\hspace*{\fill}

\textit{2.}
给出\(T \in L(C^2)\)满足\(0\)是其唯一特征值,但奇异值为\(0,5\).

\textit{Proof}
令\(T(z_1,z_2)=(0,5z_1)\),则\(0\)是其唯一特征值.

又\(T^*(z_1,z_2)=(5z_2,0)\),则\(T^*T(z_1,z_2)=(25z_1,0)\),即\(T^*T(1,0)=(25,0),T^*T(0,1)=(0,0)\).

\hspace*{\fill}

\textit{3.}
设\(T \in L(V)\).求证:存在等距算子\(S \in L(V)\)满足\(T=\sqrt{TT^*}S\).

\textit{Proof}
利用\textit{Theorem 7.93*}的结论,\(T=S\sqrt{T^*T}\),从而\(T^*=\sqrt{T^*T}S^*\).

将\(T^*\)替换成\(T\),\(S^*\)替换成\(S\),即\(T=\sqrt{TT^*}S\).

\hspace*{\fill}

\textit{8.}
设\(T,S,R \in L(V)\),\(S\)是等距算子.\(R\)是满足\(T=SR\)的正算子.求证:\(R=\sqrt{T^*T}\).

\textit{Proof}
\(T^*=(SR)^*=R^*S^*\),故\(T^*T=R^*(S^*S)R=R^*R=R^2\),据定理7.35有\(R=\sqrt{T^*T}\).

\hspace*{\fill}

\textit{9.}
设\(T \in L(V)\),证明:\(T\)是可逆算子当且仅当其极分解是唯一的.

\textit{Proof}
根据\textit{Lemma 7.64*},\(T\)是可逆算子当且仅当\(T^*T,\sqrt{T^*T}\)是可逆算子.

必要性:设\(T=S_1\sqrt{T^*T}=S_2\sqrt{T^*T}\).由于\(\sqrt{T^*T}=\sqrt{T^*T}^*\),故\(T^*=\sqrt{T^*T}S_1^*\).

于是\(T^*T=\sqrt{T^*T}\sqrt{T^*T}=\sqrt{T^*T}(S_1^*S_2)\sqrt{T^*T}\).

\(\sqrt{T^*T}\)是可逆算子,故两侧的\(\sqrt{T^*T}\)可以被消去,得到\(S_1^*S_2=I\).

由于\(S_1,S_2\)都是等距算子,故\(S_1^*=S^{-1},S_2^*=S^{-2},S_1=S_2\),反之亦然.

充分性:设\(T\)不可逆,则\(\Ker \sqrt{T^*T}=\Ker T \ne \{0\}\),取其中任意向量\(v\).

\(S\)作用于\(v\)的结果可以是任意的,从而导致了\(S\)的不唯一,矛盾.

\hspace*{\fill}

\textit{11.}
设\(T \in L(V)\).证明:\(T\)和\(T^*\)有相同的奇异值.

\textit{Proof}
由\(T^*T\)是自伴算子,\(E(\lambda,\sqrt{T^*T}) \ne \{0\} \Leftrightarrow E(\lambda,\sqrt{T^*T}^*)\)
\(=E(\lambda,\sqrt{TT^*}) \ne \{0\}\).

因此\(\lambda\)是\(T\)的奇异值和\(\lambda\)是\(T^*\)的奇异值是等价的.

\hspace*{\fill}

\textit{15.}
设\(S \in L(V)\).求证:\(S\)是等距算子等价于\(S\)的所有奇异值均为\(1\).

\textit{Proof}
充分性:\(\sqrt{S^*S}=\sqrt{S^{-1}S}=I\),显然其特征值均为\(1\),故\(S\)的所有奇异值均为\(1\).

必要性:沿用\textit{Theorem 7.70*}证明中的记号,则\(Sv=\sum_{i=1}^n \ip{v}{e_i}f_i\).

沿用\textit{Theorem 7.93*}的证明即可得到\(\norm{Sv}=v\).

\newpage

\textit{17.}
沿用\textit{Theorem 7.70*}证明中的记号,设\(T \in L(V)\)满足\(Tv=\sum_{i=1}^n s_i\ip{v}{e_i}f_i\),求证:

(\textit{a})\(T^*v=\sum_{i=1}^n s_i\ip{v}{f_i}e_i\). \quad
(\textit{b})\(T^*Tv=\sum_{i=1}^n s_i^2 \ip{v}{e_i}e_i\). 

(\textit{c})\(\sqrt{T^*T}v=\sum_{i=1}^n s_i\ip{v}{e_i}e_i\). \quad
(\textit{d})若\(T\)可逆,则\(T^{-1}v=\sum_{i=1}^n s_i^{-1}\ip{v}{f_i}e_i\).

\textit{a.Proof}
设\(\forall v,w \in V\),利用\(\ip{Tv}{w}=\ip{v}{T^*w}\),有
    \begin{align*}
        \ip{Tv}{w}&=\ip{\sum_{i=1}^n s_i\ip{v}{e_i}f_i}{w}=\sum_{i=1}^n s_i\ip{v}{e_i}\ip{f_i}{w} \\
        &=\sum_{i=1}^n \ip{v}{\ol{s_i\ip{f_i}{w}}e_i}=\sum_{i=1}^n \ip{v}{s_i\ip{w}{f_i}e_i}=\ip{v}{T^*w}
    \end{align*}
\textit{b.Proof}
\(T^*(Tv)=T^*\sum_{i=1}^n s_i \ip{v}{e_i}f_i=\sum_{i=1}^n s_i \ip{v}{e_i}T^*f_i=\sum_{i=1}^n s_i^2 \ip{v}{e_i}e_i\).

\textit{c.Proof}
令\(Rv=\sum_{i=1}^n s_i\ip{v}{e_i}e_i\),验证\(R\)是\(T^*T\)的平方根即可.
    \begin{align*}
        R^2v=\sum_{i=1}^n s_i\ip{\sum_{i=1}^n s_i\ip{v}{e_i}e_i}{e_i}e_i
        =\sum_{i=1}^n s_i (\sum_{i=1}^n s_i\ip{v}{e_i}\ip{e_i}{e_i})e_i
        =\sum_{i=1}^n s_i^2\ip{v}{e_i}e_i
    \end{align*}
容易验证\(R\)是正算子,故根据定理7.36,\(R=\sqrt{T^*T}\).

\textit{d.Proof}
令\(w=\sum_{i=1}^n s_i^{-1}\ip{v}{f_i}e_i\),则
    \begin{align*}
        Tw=\sum_{i=1}^n s_i^{-1}\ip{v}{f_i}Te_i=\sum_{i=1}^n s_i^{-1}\ip{v}{f_i}s_if_i
        =\sum_{i=1}^n \ip{v}{f_i}f_i=v
    \end{align*}
因此\(Tw=v,w=T^{-1}v=\sum_{i=1}^n s_i^{-1}\ip{v}{f_i}e_i\).

\hspace*{\fill}

\textit{18.}
设\(T \in L(V)\).令\(s_{\min},s_{\max}\)分别指代\(T\)的最小和最大奇异值,\(\lambda\)是其特征值.求证:

(\textit{a})\(\forall v \in V,s_{\min}\norm{v} \leq \norm{Tv} \leq s_{\max}\norm{v}\). \quad
(\textit{b})\(s_{\min} \leq \abs{\lambda} \leq s_{\max}\).

\textit{a.Proof}
设\(Tv=\sum_{i=1}^n s_i\ip{v}{e_i}f_i\),则
    \begin{align*}
        s_{\min}^2=s_{\min}^2 \sum_{i=1}^n \norm{\ip{v}{e_i}f_i}^2 \leq \norm{Tv}^2
        =\sum_{i=1}^n \norm{s_i\ip{v}{e_i}f_i}^2 \leq s_{\max}^2 \sum_{i=1}^n \norm{\ip{v}{e_i}f_i}^2=s_{\max}^2
    \end{align*}
因此\(s_{\min}\norm{v} \leq \norm{Tv} \leq s_{\max}\norm{v}\).

\textit{b.Proof}
设\(v\)是\(\lambda\)的特征向量,
则\(s_{\min}\norm{v}\leq \norm{Tv}=\norm{\lambda v}=\abs{\lambda}\norm{v} \leq s_{\max}\norm{v}\).

于是\(s_{\min} \leq \abs{\lambda} \leq s_{\max}\).

