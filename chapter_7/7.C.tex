\textit{1.}
证明或给出反例:若\(T \in L(V)\)是自伴算子且\(e_1,\cdots,e_n\)是\(V\)的规范正交基,

且满足\(\forall i=1,\cdots,n,\ip{Te_i}{e_i} \geq 0\),则\(T\)是正算子.

\textit{Proof}
反例:令\(V=R^2,T(z_1,z_2)=(z_1,-z_2)\),内积为欧式内积.显然\(T\)是自伴算子.

\(e_1=\dfrac{1}{\sqrt{2}}(1,-1),e_2=\dfrac{1}{\sqrt{2}}(1,1)\)是一组规范正交基.
\(\ip{Te_1}{e_1}=0,\ip{Te_2}{e_2}=0\).

然而\(\ip{T(0,1)}{(0,1)}=-1<0\),即\(T\)不是正算子.

\hspace*{\fill}

\textit{2.}
若\(T \in L(V)\)是正算子且\(\exists v,w \in V,Tv=w,Tw=v\).证明:\(v=w\).

\textit{Proof}
由\(T^2v=v,T^2w=w\),故有\(v,w \in E(1,T^2)\).

由于\(T\)是\(T^2\)的唯一正平方根,故其特征值必为\(T^2\)的算术平方根,即\(1\).

由于\(E(1,T^2)=E(1,T)\),故\(v,w \in E(1,T)\),得\(Tv=v,Tw=w\),即\(w=v\).

\hspace*{\fill}

\textit{6.}
若\(T \in L(V)\)是正算子.证明:\(\forall k \in N^*,T^k\)是正算子.

\textit{Proof}
\(T\)是自伴算子,由谱定理给出\(V\)的一组规范正交特征基\(e_1,\cdots,e_n\).

得到\(T,T^k\)在这组基下的矩阵\(M(T)=\diag(\lambda_1,\cdots,\lambda_n),M(T^k)=\diag(\lambda_1^k,\cdots,\lambda_n^k)\).

由\(\forall i=1,\cdots,n,\lambda_i \geq 0\)得到\(\lambda_i^k \geq 0\).
根据定理7.35(\textit{a})(\textit{b}),\(T^k\)是正算子.

\hspace*{\fill}

\textit{7.}
设\(T \in L(V)\)是正算子.证明:\(T\)可逆等价于\(\forall v \ne 0 \in V,\ip{Tv}{v}>0\).

\textit{Proof}
\(\forall v \ne 0 \in V,\ip{Tv}{v}>0\)等价于\(\forall v \ne 0 \in V,Tv \ne 0,\Ker T=\{0\}\),即\(T\)可逆.

\hspace*{\fill}

\textit{8.}
设\(T \in L(V),\forall u,v \in V\).定义\(\ip{u}{v}_T\)为\(\ip{u}{v}_T=\ip{Tu}{v}\).

求证:\(\ip{u}{v}_T\)是\(V\)上的内积等价于\(T\)在原内积定义下是可逆正算子.

\textit{Proof}
充分性:首先验证\(\ip{u}{v}_T\)的左变元线性.

\(\ip{\lambda u_1+\mu u_2}{v}_T=\ip{T(\lambda u_1+\mu u_2)}{u}\)
\(=\lambda \ip{Tu_1}{v}+\mu \ip{Tu_2}{v}=\lambda \ip{u_1}{v}_T+\mu \ip{u_2}{v}_T\).

再验证其共轭对称性.\(\ip{v}{Tu}=\ol{\ip{Tu}{v}}=\ol{\ip{u}{v}_T}=\ip{v}{u}_T=\ip{Tv}{u}\).

因此若要满足共轭对称性,\(T\)必须为自伴算子.

验证正定性.\(\ip{u}{u}_T=\ip{Tu}{u} \geq 0\),故\(T\)必为正算子.

令\(\ip{Tu}{u}=0\)推出\(u=0\),即\(u \ne 0\)推出\(\ip{Tu}{u}>0\),根据\textit{7.C.7},\(T\)是可逆正算子.

必要性的证明路径与之相同,容易验证.

\newpage

\textit{10.}
设\(V\)是有限维内积空间且\(S \in L(V)\),证明以下命题等价:

(\textit{a})\(S\)是等距变换;

(\textit{b})\(\forall u,v \in V,\ip{S^*u}{S^*v}=\ip{u}{v}\);

(\textit{c})对于\(V\)的任意一组规范正交基\(e_1,\cdots,e_n\),\(S^*e_1,\cdots,S^*e_n\)也是\(V\)的一组规范正交基.

\textit{Proof}
先证明\(a \Rightarrow b\).由定理7.42(\textit{a})(\textit{g})得到\(S^*\)是等距算子.根据\textit{6.A.20},有
    \begin{align*}
        \ip{S^*u}{S^*v}&=\dfrac{1}{4}(\norm{S^*u+S^*v}^2-\norm{S^*u-S^*v}^2+\norm{S^*u+iS^*v}^2i-\norm{S^*u-iS^*v}^2i) \\
        &=\dfrac{1}{4}(\norm{S^*(u+v)}^2-\norm{S^*(u-v)}^2+\norm{S^*(u+iv)}^2i-\norm{S^*(u-iv)}^2i) \\
        &=\dfrac{1}{4}(\norm{u+v}^2-\norm{u-v}^2+\norm{u+iv}^2i-\norm{u-iv}^2i)=\ip{u}{v}
    \end{align*}
实内积空间的情况用\textit{6.A.19}证明,\(a \Rightarrow b\)成立.

再证明\(b \Rightarrow c\).\(\forall i,j=1,\cdots,n,\ip{S^*e_i}{S^*e_j}=\ip{e_i}{e_j}=\delta_{i,j}\).
因此\(S^*e_1,\cdots,S^*e_n\)是规范正交基.

最后证明\(c \Rightarrow a\).设\(e_1,\cdots,e_n\)是\(V\)的一组规范正交基.根据定理6.30,有

\(\forall v=\sum_{i=1}^n \ip{v}{e_i}e_i \in V,\norm{v}^2=\sum_{i=1}^n \ip{v}{e_i}^2\),
\(\norm{Sv}^2=\sum_{i=1}^n \ip{Sv}{e_i}^2=\sum_{i=1}^n \ip{v}{S^*e_i}^2\).

由于\(S^*e_1,\cdots,S^*e_n\)是\(V\)的规范正交基,故\(\sum_{i=1}^n \ip{v}{S^*e_i}^2=\norm{v}^2=\norm{Sv}^2\).

从而\(S\)是等距算子,\(a \Rightarrow b \Rightarrow c \Rightarrow a\),等价性得证.

\hspace*{\fill}

\textit{11.}
设\(V\)是有限维内积空间,\(\mydim V=n\).\(T_1,T_2 \in L(V)\)是自伴算子且拥有相同的\(n\)个特征值.

求证:存在等距算子\(S \in L(V)\),使得\(T_1=S^*T_2S\).

\textit{Proof}
根据谱定理分别给出\(T_1,T_2\)的一组规范正交特征基\(e_1,\cdots,e_n,f_1,\cdots,f_n\).

从而\(M(T_1,e_1,\cdots,e_n)=M(T_2,f_1,\cdots,f_n)=\diag(\lambda_1,\cdots,\lambda_n)\).

定义\(S \in L(V)\)为\(\forall i=1,\cdots,n,Se_i=f_i\).
由于\(\norm{Se_i}=\norm{f_i}=\norm{e_i}=1\),故\(S\)是等距算子.

根据定理7.42(\textit{a})(\textit{h}),\(S^{-1}=S^*\).于是对于\(\forall v=\sum_{i=1}^n \ip{v}{e_i}e_i\),有
    \begin{align*}
        (S^*T_2S)v&=(S^*T_2S)\sum_{i=1}^n \ip{v}{e_i}e_i=S^*T_2\sum_{i=1}^n \ip{v}{e_i}f_i
        =S^*\sum_{i=1}^n \ip{v}{e_i}\lambda_if_i \\
        &=\sum_{i=1}^n \lambda_i\ip{v}{e_i}S^{-1}f_i
        =\sum_{i=1}^n \lambda_i\ip{v}{e_i}e_i=T_1\sum_{i=1}^n \ip{v}{e_i}e_i=T_1v
    \end{align*}

\hspace*{\fill}

总结:正算子是一类特殊的自伴算子,它们的特征值均为正数;

等距算子是一类特殊的正规算子,它们的特征值的绝对值均为\(1\).

\hspace*{\fill}

有傅立叶变换背景的\textit{19*}将单独列出.

