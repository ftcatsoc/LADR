\textit{13.}
若对于\(\forall i=1,\cdots,m\),都有\(U_i\)为\(V\)的子空间,

求证:\(\bigcup_{i=1}^m U_i\)仍是\(V\)的子空间与\(\exists j=1,\cdots,m\),
使得\(\forall i=1,\cdots,m\),\(U_i \subseteq U_j\)等价.

\textit{Proof}:
充分性的证明非常显然.

若\(\forall i=1,\cdots,m\),\(U_i \subseteq U_j\),则\(\bigcup_{i=1}^m U_i=U_j\),
而\(U_j\)是\(V\)的一个子空间,证毕.

必要性:先来根据二元情况证明一个引理.

\textit{lemma}:
若\(V_1,V_2\)是\(V\)的子空间且\(V_1 \cup V_2\)仍为\(V\)的子空间,
则必然有\(V_1 \subseteq V_2\)或\(V_2 \subseteq V_1\).

\textit{Proof}:
任取\(v_1 \in V_1 \backslash V_2\)和\(v_2 \in V_2 \backslash V_1\),考虑\(v_1+v_2\).

由于\(V_1 \cup V_2\)仍为\(V\)的子空间,故有\(v_1+v_2 \in V_1\)或\(v_1+v_2 \in V_2\).

若\(v_1+v_2 \in V_1\),则\(v_2=(v_1+v_2)+(-v_1) \in V_1\),与\(v_2 \notin V_1\)矛盾;

若\(v_1+v_2 \in V_2\),则\(v_1=(v_1+v_2)+(-v_2) \in V_2\),与\(v_1 \notin V_2\)矛盾.

因此必有\(V_1 \subseteq V_2\)或\(V_2 \subseteq V_1\).

现在使用数学归纳法应用该引理,先考虑\(i=1\)的情况.令\(V_1=U_1,V_2=\bigcup_{i=2}^m U_i\).

则必有\(1^{\circ}\bigcup_{i=1}^m U_i \subseteq U_1\)或\(2^{\circ}U_1 \subseteq \bigcup_{i=j+1}^m U_i\).

若为\(1^{\circ}\)则\(U_1\)即为所求,若为\(2^{\circ}\)则\(\bigcup_{i=2}^m U_i\)是\(V\)的子空间,归纳继续.

设\(i=j\)时归纳继续,即\(U_j \subseteq \bigcup_{i=j+1}^m U_i\).
则当\(i=j+1\)时,令\(V_1=U_{j+1},V_2=\bigcup_{i=j+2}^m U_i\),

仍有\(\bigcup_{i=j+2}^m U_i \subseteq U_{j+1}\)或\(U_{j+1} \subseteq \bigcup_{i=j+2}^m U_i\),
即\(U_{j+1}\)是所求子集,或者归纳可以继续.

由于子空间个数有限,因此进程一定可以在某一步结束,下证结束时找到的\(U_j\)即为所求.

不妨假设第\(j\)步找到了满足条件的\(U_j\),即\(U_j\)满足\(\bigcup_{i=j+1}^m U_i\subseteq U_j\).

此时回到第\(j-1\)步,由于该步必定出现了\(2^{\circ}\),
因此有\(U_{j-1} \subseteq \bigcup_{i=j}^m U_i=U_j\),即\(U_{j-1} \subseteq U_j\).

随即\(\bigcup_{i=j-1}^m U_i=U_j\),进而\(U_{j-2} \subseteq \bigcup_{i=j-1}^m U_i=U_j\).
以此类推,\(\forall i=1,\cdots,j-1\),\(U_i\subseteq U_j\).

结合\(\bigcup_{i=j+1}^m U_i \subseteq U_j\),得\(\forall i=1,\cdots,m,U_i \subseteq U_j\),证毕.

