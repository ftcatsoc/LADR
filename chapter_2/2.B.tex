\textit{8.}
设\(U\)和\(W\)都是\(V\)的子空间,且满足\(V=U \oplus W\).

\(u_1,\cdots,u_m,w_1,\cdots,w_n\)分别是\(U,W\)的一组基.
求证:\(u_1,\cdots,u_m,w_1,\cdots,w_n\)是\(V\)的一组基.

\textit{Proof}
先证明\(u_1,\cdots,u_m,w_1,\cdots,w_m\)线性无关.

设\(\exists a_1,\cdots,a_m,b_1,\cdots,b_n,\sum_{i=1}^m a_iu_i+\sum_{i=1}^n b_iw_i=0\),
即\(v=\sum_{i=1}^m a_iu_i=-\sum_{i=1}^n b_iw_i\).

这表明\(U\)中的某元素与\(W\)中的某元素相等,即\(v\in U\cap W\).

而\(V=U\oplus W\),即\(V\cap W=\{0\}\),得\(\sum_{i=1}^m a_iu_i=-\sum_{i=1}^n b_iw_i=0\).

由\(u_1,\cdots,u_m\)和\(w_1,\cdots,w_m\)分别线性无关,有\(u_1=\cdots=u_m=w_1=\cdots=w_n=0\),得证.

再证\(V=\myspan(u_1,\cdots,u_m,w_1,\cdots,w_n)\),且\(\myspan(u_1,\cdots,u_m,w_1,\cdots,w_n) \subseteq V\)是显然的.

由\(V=U\oplus W\),得\(\forall v\in V,\exists u \in \myspan(u_1,\cdots,u_m),w \in \myspan(w_1,\cdots,w_n),v=u+w\).

因此\(\forall v \in V,v \in \myspan (u_1,\cdots,u_m,w_1,\cdots,w_n)\),
即\(V \subseteq \myspan(u_1,\cdots,u_m,w_1,\cdots,w_n)\).

结合\(\myspan(u_1,\cdots,u_m,w_1,\cdots,w_n) \subseteq V\),
得到\(V=\myspan (u_1,\cdots,u_m,w_1,\cdots,w_n)\),证毕.


